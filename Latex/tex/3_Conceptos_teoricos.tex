\capitulo{3}{Conceptos teóricos}

Para construir un modelo capaz de capturar con precisión las características distintivas de las superficies objetivo, primero se debe comprender el origen y la estructura de la señal recibida. En este apartado se introducen algunos conceptos fundamentales del sistema de radar.

\section{Radar Acconeer}

El radar utilizado en el proyecto está fabricado por Acconeer llamado A111. Es un radar de 60GHz basado en impulsos tecnología de radar coherente (PCR\footnote{Pulsed
Coherent Radar}) totalmente integrado en un pequeño chip de 29 mm2.
Esto permitirá una fácil integración en cualquier dispositivo portátil impulsado por batería.

\imagen{radar}{Radar A111}

Aplicaciones:
\begin{itemize}
	\item Mediciones de distancia de alta precisión con mm de precisión y alta tasa de actualización.
	\item Detección de movimiento.
	\item Detección de velocidad.
	\item Detección de material.
	\item Seguimiento de objetos de alta precisión como el control de gestos.
	\item Seguimiento de alta precisión de objetos 3D.
	\item Control de los signos vitales, como la respiración y la frecuencia del pulso.
\end{itemize}

\imagen{fig_block_diagram}{Diagrama de bloques del sensor A111}

La figura 3.2 muestra un diagrama de bloques del sensor A111. La señal se transmite desde la antena Tx y es recibida por la antena Rx, ambas integradas en la capa superior del sustrato del paquete A111. Además de la radio mmWave, el sensor consta de administración de energía y control digital, cuantificación de señales, memoria y un circuito de temporización.

El sensor se puede ejecutar en uno de estos servicios básicos:
\tablaSmall{Servicios del radar A111}{l c c}{serviciosa111}
{\multicolumn{1}{l}{Servicio} & Tipo de dato & Ejemplo de uso \\}{ 
Envelope & Amplitud & Distancia absoluta y
presencia estática \\
IQ & Amplitud y fase & Detección de obstáculos, respiración y distancia relativa \\
Sparse & Amplitud instantánea & Velocidad, detección de presencia y detección de gestos \\
} 
