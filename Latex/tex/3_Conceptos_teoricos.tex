\capitulo{3}{Conceptos teóricos}

Para construir un modelo capaz de capturar con precisión las características distintivas de las superficies objetivo, primero se debe comprender el origen y la estructura de la señal recibida. En este apartado se introducen algunos conceptos fundamentales del sistema de radar.
\section{Teledetección}

Teledetección o detección a distancia (detección remota) es la técnica que permite recopilar información a distancia de objetos sin que exista un contacto material. Para que esto sea posible tendremos una interacción entre los objetos y un sensor situado en una plataforma.

La distancia a que debe estar situado un sensor para ser remoto puede variar desde pocos decímetros hasta miles de kilómetros.

La teledetección es un flujo de radiación emitido por los objetos o materiales hacia un radar o sensor. El origen del flujo puede venir de:

\begin{itemize}
\item Radiación solar reflejada por los objetos - Teledetección pasiva
\item Radiación terrestre emitida por los objetos - Teledetección pasiva
\item Radiación emitida por el sensor y reflejada por los objetos (radar) - Teledetección activa
\end{itemize}

Se llama teledetección pasiva a la técnica aplicada por los sensores que miden las variaciones de la energía procedente de los objetos sin intervenir en el campo natural y se denomina teledetección activa a aquellos que generan un campo de energía artificial, utilizado para registrar y medir el efecto que en él producen los objetos.
 
Las características técnicas del sensor influye en la calidad de los datos y en la posibilidad de recibir información en distintas longitudes de onda. 

\imagen{sitemas_globales}{Sistemas de observación global.}

En la figura 3.1 vemos la globalización actual de la teledetección utilizada para obtener información precisa, actualizada y de fácil acceso empleada para diversas utilidades y finalidades como pueden ser:
\begin{itemize}
\item Imágenes territoriales
\item Motorización de las mareas
\item Análisis de materiales terrestres
\end{itemize}

\section{Radar Acconeer}

El radar utilizado en el proyecto está fabricado por Acconeer llamado A111. Es un radar de 60GHz basado en impulsos tecnología de radar coherente (PCR\footnote{Pulsed
Coherent Radar}) totalmente integrado en un pequeño chip de 29 mm2.
Esto permitirá una fácil integración en cualquier dispositivo portátil impulsado por batería.

\imagen{radar}{Radar A111.}

Aplicaciones:
\begin{itemize}
	\item Mediciones de distancia de alta precisión con mm de precisión y alta tasa de actualización.
	\item Detección de movimiento.
	\item Detección de velocidad.
	\item Detección de material.
	\item Seguimiento de objetos de alta precisión como el control de gestos.
	\item Seguimiento de alta precisión de objetos 3D.
	\item Control de los signos vitales, como la respiración y la frecuencia del pulso.
\end{itemize}

\imagen{figBlockDiagram}{Diagrama de bloques del sensor A111.}

La figura 3.2 muestra un diagrama de bloques del sensor A111. La señal se transmite desde la antena Tx y es recibida por la antena Rx, ambas integradas en la capa superior del sustrato del paquete A111. Además de la radio mmWave, el sensor consta de administración de energía y control digital, cuantificación de señales, memoria y un circuito de temporización.

El sensor se puede ejecutar en uno de los siguientes servicios básicos de la tabla 3.1.

\tablaSmall{Servicios del radar A111}{l c c}{serviciosa111}
{\multicolumn{1}{l}{Servicio} & Tipo de dato & Ejemplo de uso \\}{ 
Envelope & Amplitud & Distancia absoluta y
presencia estática \\
IQ & Amplitud y fase & Detección de obstáculos, respiración y distancia relativa \\
Sparse & Amplitud instantánea & Velocidad, detección de presencia y detección de gestos \\
} 


\section{Servicio IQ}

El servicio IQ utiliza la coherencia de fase del radar pulsado Acconeer para producir componentes estables en fase y en cuadratura. Este servicio se puede utilizar para la detección de presencia frente al sensor, la detección de la frecuencia respiratoria, la detección de obstáculos y , en nuestro caso, para diferenciar materiales.

Los componentes en fase y en cuadratura se representan como valores complejos, lo que genera un conjunto complejo de N\textsubscript{D} muestras representadas como x[d], dónde d es el índice de demora de la muestra.

Los datos obtenidos a través del servicio IQ proporcionan un método para examinar la reflectividad a diferentes distancias del sensor de radar.


\section{Aprendizaje automático}

El aprendizaje automático o machine learning es un tipo de inteligencia artificial (AI),
consiste en programar una computadora para que mejore en la realización de una tarea a partir de datos de ejemplo o de la experiencia.
