\capitulo{1}{Introducción}

La tecnología de radar existe desde la década de 1930 de la mano de Watson-Watt, el término radar proviene del acrónimo inglés RAdio Detection And Ranging, el uso típico de los radares de radiofrecuencia se basa en medir distancias, altitudes, direcciones y velocidades de objetos, además de tener un uso muy reconocido el cual es el mapa de navegación de los barcos.

Durante los últimos años han surgido nuevas áreas de aplicación las cuales plantean diferentes desafíos para la humanidad, dado que una aplicación es monitorizar los signos vitales, reconocimiento de gestos, entre otros, dado que los radares últimamente se han vuelto más baratos, en gran parte debido a su adopción en la industria automotriz, lo que convierte a estos dispositivos en una opción atractiva en una amplia gama de aplicaciones de bajo costo.

El presente trabajo está basado en documentar, implementar, tratando de demostrar que mediante el uso de un radar de 60GHz fabricado por Acconeer es posible el reconocimiento de diferentes materiales, por lo cual se plantea el implementar un procedimiento el cual ayude a la obtención de las diferentes características, utilizando algoritmos de aprendizaje automático, dado que se busca realizará un registro de tres tipos de materiales (cartón, plástico y cristal) para crear un modelo de Inteligencia Artificial para conseguir un correcto funcionamiento del radar.
