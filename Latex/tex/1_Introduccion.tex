\capitulo{1}{Introducción}

La tecnología de radar existe desde la década de 1930 de la mano de Watson-Watt. El término radar proviene del acrónimo inglés RAdio Detection And Ranging. Aplicaciones típicas de los radares de radiofrecuencia incluyen la medida de distancias, altitudes, direcciones y velocidades de objetos. Un ejemplo muy conocido de uso es para «navegación de barcos».

Durante los últimos años han surgido nuevas áreas de aplicación para este tipo de radares. Actualmente se están empleando en monitorizar signos vitales, reconocimiento de gestos, etc. Su creciente aplicación se explica, en parte, por la disminución en su coste de adquisición, por (también, en parte) su implantación en industrias como la del automóvil, convirtiendo estos dispositivos en una opción atractiva en una amplia gama de aplicaciones de bajo coste.

En el presente proyecto se ha documentado, implementado y demostrado el uso de un radar de 60GHz fabricado por \textit{Acconeer}, para el reconocimiento de diferentes materiales u objetos. Mediante el uso de métodos de aprendizaje, se obtendrán diferentes características de tres tipos de materiales, que permitirán clasificarlos de forma automática.

