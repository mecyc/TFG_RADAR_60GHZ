\capitulo{1}{Introducción}

La tecnología de radar existe desde la década de 1930 de la mano de Watson-Watt.
El término radar proviene del acrónimo inglés RAdio Detection And Ranging.

El uso típico de los radares de radiofrecuencia se basa en  medir distancias, altitudes, direcciones y velocidades de objetos. Un uso muy reconocido es el mapa de navegación de los barcos.

Durante los últimos años han surgido nuevas áreas de aplicación que plantean
diferentes desafíos. Una aplicación es monitorizar los signos vitales, reconocimiento de gestos, entre otros.

Los radares últimamente se han vuelto más baratos, en gran parte debido a su
adopción en la industria automotriz, lo que convierte a estos dispositivos en una
opción atractiva en una amplia gama de aplicaciones de bajo costo.


Este trabajo se centra en documentar y demostrar el uso de un radar de 60GHz fabricado por Acconeer. Usando un procedimiento de extracción de características y aprendizaje automático.
Para ello se realizará un registro de tres tipos de materiales (cartón, plástico y cristal) y se creará un modelo de IA para conseguir un correcto funcionamiento del radar.
