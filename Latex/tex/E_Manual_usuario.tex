\apendice{Documentación de usuario}

\section{Introducción}

En este apartado se explican los requerimientos de la aplicación para ser ejecutada, el proceso de instalación y el manual
de usuario con las indicaciones para utilizar correctamente la aplicación.

\section{Requisitos de usuarios}

En esta sección se indican los requisitos para utilizar la aplicación.

Si vamos a clasificar lecturas sin radar necesitamos:
\begin{itemize}
\item Lecturas guardadas en el equipo en formato \textit{.npy}
\end{itemize}

Si deseamos realizar lecturas desde radar necesitamos el siguiente hardware:
\begin{itemize}
\item Raspberry Pi 4
\item Radar A111
\item Placa XR112
\item Cable flexible para XR112
\item Tarjeta SD
\item Teclado USB
\item Ratón USB
\item Monitor con HDMI
\item Cable HDMI
\item Adaptador mini HDMI a HDMI
\end{itemize}

Y además necesitamos la siguiente configuración:
\begin{itemize}
\item Equipo conectado a red \textit{A}.
\item Radar conectado a la misma red \textit{A}.
\item Antes de realizar una lectura por radar necesitamos conectarle a la red eléctrica unos 20 segundos antes de iniciar la lectura.
\end{itemize}


\subsection{Instalación en \textit{Windows}}

La siguiente instalación se ha realizado en un equipo con las siguientes características:
\begin{itemize}
\item[•] \textbf{SISTEMA OPERATIVO:} \textit{Microsoft Windows 10 Pro}
\item[•] \textbf{PROCESADOR:} \textit{Intel(R) Core(TM) i7-4720HQ CPU @ 2.60GHz}
\item[•] \textbf{MEMORIA RAM:} 12 GB
\item[•] \textbf{TARJETA GRÁFICA:} \textit{NVIDIA GeForce 930M}
\end{itemize}

En realidad la aplicación \textit{RadarWave} no hace falta que se instale, pero si necesitamos instalar las librerías utilizadas por el programa.

Antes de proceder con la instalación de las librerías vamos a verificar si tenemos instalado Python, de no ser así procedemos a su instalación.
\begin{itemize}
\item Descargamos el instalador desde la web de \href{https://www.python.org/downloads/}{\textit{Python}}.
\item Ejecutamos el archivo descargado.
\item En la pantalla de de inicio de la instalación marcamos la opción de añadir \textit{Python} al \textit{PATH}, si no luego se tendrá que hacer manualmente.
\item Pulsamos en instalar ahora (\textit{install now})

\imagen{path}{Instalación de \textit{Python}.}

\end{itemize}

Una vez instalado \textit{Python} lo primero que haremos será descargar y descomprimir el fichero \textit{$RadarWave\_v1.0.zip$}, se puede descargar desde \url{https://github.com/mecyc/TFG_RADAR_60GHZ/blob/main/RadarWave_v1.0.zip} 

Procedemos a instalar las librerías, están incluidas en el archivo \textit{requirements.txt}. Primero instalamos \textit{setuptools} que facilita el empaquetado de proyectos de \textit{Python} y a continuación el resto, para ello debemos abrir la consola y lanzar las siguientes instrucciones.

\begin{verbatim}
python -m pip install -U --user setuptools wheel
python -m pip install -U --user -r requirements.txt
\end{verbatim}

\imagen{requirements}{Instalar librerías.}

Finalmente se debe instalar la librería de \textit{Acconeer}, para ello descargamos del repositorio del TFG la carpeta \href{https://github.com/mecyc/TFG_RADAR_60GHZ/tree/main/acconeer-python-exploration}{\textit{acconer-python-exploration}}. Una vez descargada abrimos el terminal dentro de la carpeta y ejecutamos la siguiente orden.\cite{Acconeer2021}

\begin{verbatim}
python -m pip install -U --user .
\end{verbatim}

\imagen{instalar_libacconer}{Instalar librería \textit{Acconeer}.}

Una vez instaladas las librerías procedemos a ejecutar el programa haciendo doble clic en el fichero \textit{RadarWave.py} donde está el programa.

Programa ejecutado y probado en \textit{Windows}:

\imagen{radarwaveWindows}{\textit{RadarWave} en \textit{Windows}.}

\subsection{Instalación en \textit{Linux}}

La siguiente instalación se ha realizado en un equipo virtualizado en \textit{Oracle VM VirtualBox 6.1} con las siguientes características:
\begin{itemize}
\item[•] \textbf{SISTEMA OPERATIVO:} \textit{Ubuntu 22.04 LTS}
\item[•] \textbf{PROCESADOR:} \textit{Intel(R) Core(TM) i7-4720HQ CPU @ 2.60GHz}
\item[•] \textbf{MEMORIA RAM:} 4 GB
\end{itemize}

En realidad la aplicación \textit{RadarWave} no hace falta que se instale, pero si necesitamos instalar las librerías utilizadas por el programa.

El sistema operativo que utilizamos es \textit{Ubuntu}, una distribución {Linux}, por lo que \textit{Python} está integrado por defecto y no hace falta su instalación. Se recomienda tener actualizado \textit{Python} a la última versión.

Antes de instalar las librerías necesarias instalamos \textit{pip}, es un sistema de gestión de paquetes que nos ayuda a instalar las librerías, y \textit{setuptools}, facilita el empaquetado de proyectos de Python.

\begin{verbatim}
sudo apt install python3-pip
pip3 install setuptools-rust
\end{verbatim}

Una vez terminada la instalación procedemos a descargar y descomprimir el fichero \textit{$RadarWave\_v1.0.zip$}, se puede descargar desde \url{https://github.com/mecyc/TFG_RADAR_60GHZ/blob/main/RadarWave_v1.0.zip}

A continuación se debe instalar las librerías que usa la interfaz desarrollada. Indicar que para ejecutar el siguiente comando nos debemos situar en la dirección o \textit{PATH} donde se encuentra el archivo \textit{requirements.txt}, lo encontramos dentro del fichero que acabamos de descomprimir.

El comando sería el siguiente:
\begin{verbatim}
pip3 install -r requirements.txt
\end{verbatim}

Si existiera algún problema al instalar la librería de \textit{Tkinter} (tk) se deben lanzar los siguientes comandos:
\begin{verbatim}
sudo apt-get install python3-tk
sudo apt-get install python3-pil python3-pil.imagetk
\end{verbatim}

Finalmente se debe instalar la librería de \textit{Acconeer}, para ello descargamos del repositorio del TFG la carpeta \href{https://github.com/mecyc/TFG_RADAR_60GHZ/tree/main/acconeer-python-exploration}{\textit{acconer-python-exploration}}. Una vez descargada abrimos el terminal dentro de la carpeta y ejecutamos la siguiente orden.

\begin{verbatim}
python3 setup.py install
\end{verbatim}

Si queremos ejecutar \textit{RadarWave} en \textit{Ubuntu} deber ser lanzando el siguiente comando en la misma ruta del archivo \textit{RadarWave.py}.

\begin{verbatim}
python3 RadarWave.py
\end{verbatim}


Programa ejecutado y probado en \textit{Ubuntu}:

\imagen{radarwaveUbuntu}{\textit{RadarWave} en \textit{Ubuntu}.}

\section{Manual del usuario}

En esta sección indicaremos al usuario cómo usar la aplicación.

Para comenzar debemos iniciar la aplicación abriendo el archivo \textit{RadarWave.py}.

Nos muestra la siguiente pantalla:

\imagen{radarwave_manual}{Pantalla principal RadarWave.}

El uso de la aplicación es muy sencillo, tenemos cuatro botones el la parte superior, de izquierda a derecha son:
\begin{itemize}
\item 1. Abrir archivo de lectura
\item 2. Iniciar lectura por radar
\item 3. Clasificar la lectura
\item 4. Guardar datos de lectura.
\end{itemize}

Si pulsamos en el botón 1 se abre el explorador de archivos para seleccionar una lectura a clasificar.

Abrimos el fichero y ya tenemos los datos dentro de la aplicación, para clasificaros pulsamos el botón 3. Si lo que queremos el guardarlos pulsamos en el botón 4.

Para realizar una lectura por radar necesitamos iniciar la \textit{Raspberry} junto con el radar 20 segundos antes de lanzar la lectura, una vez hecho esto pulsamos en el botón 2. Tras esto se iluminan los botones 3 y 4. Para clasificaros pulsamos el botón 3. Si lo que queremos el guardarlos pulsamos en el botón 4.


