\capitulo{2}{Objetivos del proyecto}

El objetivo del proyecto es el desarrollo de una aplicación capaz de implementar la tecnología del radar junto con un modelo clasificador capaz de establecer una comparación entre los diferentes tipos de materiales que se estudiarán.

\section{Objetivos generales}

Los objetivos generales del proyecto se pueden describir en los siguientes puntos:
\begin{itemize}
\item[•] El principal, desarrollar una aplicación capaz de comunicarse con el radar y realizar lecturas de objetos para identificar distintos materiales.

\item[•] Documentar los diferentes elementos que componen el ensamblado y poner en funcionamiento un radar de 60 GHz fabricado por \textit{Acconeer} para que sea capaz de identificar, mediante un clasificador, el tipo de material al que pertenece un objeto.

\item[•] Crear un procedimiento capaz de extraer determinadas características de las lecturas de los objetos para entrenar un modelo de \textit{machine learning}.

\item[•] Poder generar una matriz de confusión que muestre la tasa de acierto de los diferentes materiales empleados.

\item[•] Establecer una interfaz capaz de poder ser implementada en el sistema operativo Windows y Linux, para ello la aplicación tiene que poder ser implementada en cualquier dispositivo que cuente con estos sistemas.
\end{itemize}

\section{Objetivos técnicos}

En este proyecto entran en juego diferentes campos técnicos, entre los que podemos destacar la electrónica, sistemas computacionales, física, mecánica, mecatrónica, por lo cual se presentan los siguientes objetivos técnicos:

\begin{itemize}
\item[•] Conocer e identificar las diferentes tecnologías empleadas en esta metodología particular de teledetección, describiendo qué es un radar, cuál es su operación, cuales son sus limitaciones, logrando aterrizar la idea principal del proyecto del empleo de un radar para el estudio.

\item[•] Conocer los datos técnicos del radar Acconeer el cual será implementado para establecer cuáles serán sus requerimientos en cuanto a alimentación eléctrica, conexión, interacción con los demás componentes que integran el proyecto

\item[•] Aplicar diferentes metodología de aprendizaje automático que nos permitan determina cuál es la mejor estrategia simplificando la operación de identificación de los diferentes materiales.

\item[•] Establecer las diferentes características a analizar de los materiales buscando poder diferenciarlos con el empleo del sensor radar Acconeer, para establecer la base de datos generada.

\item[•] Determinar qué lenguajes serán los óptimos para ser empleados de acuerdo con los componentes y los conocimientos obtenidos durante el análisis de la literatura que integra este proyecto.


\item[•] Establecer las bases para futuras investigaciones las cuales deseen implementar este tipo de tecnología en sus proyectos, sirviendo como referencia y apoyo a los diferentes desarrolladores.

\end{itemize}