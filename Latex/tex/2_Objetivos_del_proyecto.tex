\capitulo{2}{Objetivos del proyecto}

El objetivo de desarrollar este proyecto es el poder implementar la tecnología del radar para poder verificar su funcionamiento y operación del sensor radar por medio de una herramienta de machine learning capaz de establecer una comparación entre los diferentes tipos de materiales que se estudiaran.

\section{Objetivos generales}

Los objetivos generales del proyecto que caracterizan la propuesta que se presenta a continuación se puede describir en los siguientes puntos:
\begin{itemize}
\item[•] El objetivo principal del proyecto es poder documentar los diferentes elementos que componen el ensamblar y poner en funcionamiento un radar de 60GHz fabricado por Acconeer el cual es capaz de identificar, mediante un clasificador, el tipo de material al que pertenece un objeto.

\item[•] Crear bajo un algoritmo de machine learning un procedimiento capaz de extraer determinadas características de los materiales, con el empleo del aprendizaje automático.

\item[•] Poder generar un gráfico de probabilidades el cual sea capaz de mostrar la clasificación de los diferentes materiales empleados.

\item[•]Se establecerá una interfaz capaz de poder ser implementada en el sistema operativo Windows, buscando que la aplicación pueda ser implementada en cualquier dispositivo que cuente con este sistema.
\end{itemize}

\section{Objetivos técnicos}

Las diferentes ramas empleadas para desarrollar el proyecto hacen que sea un proyecto multidisciplinario, ya que está integrado por ramas de la electrónica, sistemas computacionales, física, mecánica, mecatrónica, por lo cual se presentan los siguientes objetivos técnicos:

\begin{itemize}
\item[•] Conocer e identificar las diferentes tecnologías empleadas en la teledetección, describiendo que es un radar, cuál es su operación, cuales son sus limitaciones, logrando aterrizar la idea principal del proyecto del empleo de un radar para el estudio.

\item[•] Conocer los datos técnicos del radar acconeer el cual será implementado para establecer cuáles serán sus requerimientos en cuanto a alimentación eléctrica, conexión, interacción con los demás componentes que integran el proyecto

\item[•] Logra establecer las diferencias entre los diferentes aprendizajes automáticos existentes para poder determinar cuál es la mejor opción de acuerdo con la información que se pretende obtener de los diferentes materiales.

\item[•] Establecer un algoritmo machine learning el cual pueda simplificar la operación de identificación de los diferentes materiales haciéndolo de una forma automática.

\item[•] Establecer las diferentes características a analizad de los materiales buscando poder diferenciarlos con el empleo del sensor radar acconeer, para establecer la base de datos generada.

\item[•] Establecer cuales técnicas y metodologías se deben implementar para poder establecer los tiempos y movimientos para realizar las revisiones y así poder obtener los resultados deseados en el proyecto

\item[•] Determinar que lenguajes serán los óptimos para ser empleados de acuerdo con los componentes y los conocimientos obtenidos durante el análisis de la literatura que integra este proyecto.

\item[•] Se establecerá cual herramienta de documentación será empleada para la elaboración de documentos que integran el proyecto pudiendo editar, gestionar y difundir la información de una forma segura.

\item[•] Establecer las bases para futuras investigaciones las cuales deseen implementar este tipo de tecnología en sus proyectos, sirviendo como referencia y apoyo a los diferentes desarrolladores.

\end{itemize}