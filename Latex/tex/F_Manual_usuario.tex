\apendice{Documentación de usuario}

\section{Introducción}

En este apartado se explican los requerimientos de la aplicación para ser ejecutada, el proceso de instalación y el manual de usuario con las indicaciones para utilizar correctamente la aplicación.

\section{Requisitos de usuarios}

En esta sección se indican los requisitos para utilizar la aplicación.

Si vamos a clasificar lecturas sin radar necesitamos:
\begin{itemize}
\item Lecturas guardadas en el equipo en formato \textit{.npy}
\end{itemize}

Si deseamos realizar lecturas desde radar necesitamos el siguiente hardware:
\begin{itemize}
\item Raspberry Pi 4
\item Radar A111
\item Placa XR112
\item Cable flexible para XR112
\item Tarjeta SD
\item Teclado USB
\item Ratón USB
\item Monitor con HDMI
\item Cable HDMI
\item Adaptador mini HDMI a HDMI
\end{itemize}

Y además necesitamos la siguiente configuración:
\begin{itemize}
\item Equipo conectado a red \textit{A}.
\item Radar conectado a la misma red \textit{A}.
\item Antes de realizar una lectura por radar necesitamos conectarle a la red eléctrica unos 20 segundos antes de iniciar la lectura.
\end{itemize}

\subsection{Instalación y ejecución en \textit{Windows}}

Consultar la sección \ref{sec:windows}.

\subsection{Instalación y ejecución en \textit{Linux}}

Consultar la sección \ref{sec:linux}.

\section{Manual del usuario}

En esta sección indicaremos al usuario cómo usar la aplicación.

Para comenzar debemos iniciar la aplicación abriendo el archivo \textit{RadarWave.py}.

Nos muestra la siguiente pantalla:

\imagen{radarwave_manual}{Pantalla principal RadarWave.}

El uso de la aplicación es muy sencillo, tenemos cuatro botones en la parte superior, de izquierda a derecha son:
\begin{itemize}
\item 1. Abrir archivo de lectura
\item 2. Iniciar lectura por radar
\item 3. Clasificar la lectura
\item 4. Guardar datos de lectura.
\end{itemize}

Si pulsamos en el botón 1 se abre el explorador de archivos para seleccionar una lectura a clasificar.

Abrimos el fichero y ya tenemos los datos dentro de la aplicación, para clasificarlos pulsamos en el botón 3. Si lo que queremos es guardarlos pulsamos en el botón 4.

Para realizar una lectura por radar necesitamos iniciar la \textit{Raspberry} junto con el radar 20 segundos antes de lanzar la lectura, una vez iniciado el radar pulsamos en el botón 2. Tras esto se iluminan los botones 3 y 4. Para clasificar el material leído pulsamos en el botón 3. Si lo que queremos es guardar la lectura pulsamos en el botón 4.

\textbf{Vídeo elaborado como demostración y manual de usuario de la aplicación RadarWave:} \url{https://youtu.be/JRan9dtaahU}


