\apendice{Plan de Proyecto Software}

\section{Introducción}
En este apartado se va a detallar la planificación temporal del proyecto,
especificando las tareas. También se realizarán estimaciones económicas y la
viabilidad legal.
\section{Planificación temporal}

Se ha decidido seguir la metodología \textit{SCRUM} para la gestión del proyecto.
Se han aplicado las siguientes características:
El desarrollo se ha dividido en iteraciones (sprints), cada una de las cuales ha tenido una duración entre 1 y 2 semanas. Durante cada sprint se irán creando y realizando las diferentes tareas correspondientes a
los objetivos fijados al inicio de cada sprint. Las tareas de cada sprint son elaboradas en la reunión que se establece al terminar la iteración anterior. Además de planificar las tareas del siguiente sprint se supervisan las tareas anteriores.

A continuación, se describen brevemente cada uno de los sprints ejecutados durante el desarrollo del proyecto.

\subsection{Sprint1 - 29 Octubre 2021 - 12 Noviembre 2021}

Sprint que da comienzo al TFG en la modalidad ONLINE. 

Indicar que el tema escogido para la realización del proyecto se hizo en abril de 2021, cursando la modalidad presencial de la carrera. La causa de que no se realizara ningún sprint en ese curso fue ocasionada por la carga lectiva más el acoplamiento de la realización de las prácticas en empresa.

Este Sprint comprende las tareas de:
\begin{itemize}
\item Familiarización con \textit{GitHub} (milestone, commits, estadisticas...)
\item Añadir \textit{Readme} al repositorio.
\item Desarrollo de la memoria del TFG.
\end{itemize}

\subsection{Sprint2 - 13 Noviembre 2021 - 26 Noviembre 2021}

En el Sprint2 se seguirá con la documentación de la memoria y se barajará crear una interfaz alternativa a la actual con alguna herramienta drag and drop para código python.

Este Sprint comprende las tareas de:

\begin{itemize}
\item Conceptos teóricos memoria.
\item Técnicas y herramientas memoria.
\item Familiarización con los anexos de la memoria.
\item Crear interfaz alternativa.
\end{itemize}

\subsection{Sprint3 - 27 Noviembre 2021 - 10 Diciembre 2021}

En el Sprint3 se seguirá con la documentación de la memoria y además se refactoriza el código de los modelos clasificatorios Random Forest, Regresión Logística y KNN

Este Sprint comprende las tareas de:

\begin{itemize}
\item Aspectos relevantes del desarrollo del proyecto.
\item Plan de proyecto software.
\item Refactorización de código fuente.
\end{itemize}

\subsection{Sprint4 - 11 Diciembre 2021 - 24 Diciembre 2021}

En el Sprint4 se envía una versión preliminar de la memoria a los tutores, tras una reunión realizada se decide modificar apartados de la memoria.

Este Sprint comprende las tareas de:

\begin{itemize}
\item Conceptos teóricos.
\item Bibliografía.
\end{itemize}

\subsection{Sprint5 - 25 Diciembre 2021 - 14 Enero 2022}

En el Sprint5 se seguirá con la documentación de la memoria y además se finaliza la interfaz gráfica que será la encargada de realizar las lecturas de los materiales.

Este Sprint comprende las tareas de:

\begin{itemize}
\item Desarrollo de interfaz.
\item Añadir «acerca de» en interfaz
\item Resumen memoria
\item Desarrollo apartados memoria
\end{itemize}
\subsection{Sprint6 - 15 Enero 2022 - 5 Febrero 2022}

En el Sprint6 se realiza la documentación de los anexos y se modifica la interfaz para implementar una barra de estado.

Este Sprint comprende las tareas de:

\begin{itemize}
\item Desarrollo de anexos.
\item Añadir barra de herramientas (interfaz).
\item Añadir barra de estado (interfaz).
\item Crear ejecutable de la aplicación.
\end{itemize}

\subsection{Sprint7 - 6 Febrero 2022 - 17 Febrero 2022}

El Sprint7 comprende el periodo final del desarrollo del proyecto.

Se han realizado la últimas tareas para la entrega del proyecto:

\begin{itemize}
\item Corrección de la documentación.
\item Reestructurar repositorio.
\item Entrega del proyecto.
\end{itemize}

Estas tareas finalmente no se cumplieron al estar el proyecto incompleto y se ha decidido retrasar la entrega para la convocatoria del siguiente curso.

\subsection{Sprint8 - 28 Octubre 2022 - 10 Noviembre 2022}

El Sprint8 comprende el periodo inicial para retomar el proyecto en la modalidad presencial de la carrera. Como se han perdido las dos convocatorias del curso 2021/22 se ha retomado el TFG para el curso 2022/23 con vistas a la convocatoria de Enero.

Para recordar el desarrollo y comenzar el proyecto desde el principio se han decidido las siguientes tareas:

\begin{itemize}
\item Lectura del proyecto.
\item Ajustar los parámetros al radar.
\item Crear una nueva biblioteca de observaciones, es decir, volver a realizar todas las lecturas de los materiales.
\item Crear nuevos modelos de entrenamiento (\textit{RandomForest, KNN, SVM})
\end{itemize}

\subsection{Sprint9 - 11 Noviembre 2022 - 24 Noviembre 2022}

En el Sprint9 se ha querido introducir nuevos clasificadores actuales e innovadores como son:

\begin{itemize}
\item \textit{auto-sklearn}
\item \textit{TabPFN}
\end{itemize}

Con estos dos clasificadores se ha querido mejorar las tasas de acierto obtenidas hasta ahora por los modelos clasificadores más convencionales.

\subsection{Sprint10 - 25 Noviembre 2022 - 8 Diciembre 2022}

En el Sprint 10 nos vamos a centrar en el desarrollo de la memoria y anexos añadiendo los nuevos cambios y modificando en aquello que sea necesario.

\subsection{Sprint11 - 9 Diciembre 2022 - 22 Diciembre 2022}

En periodo del Sprint 11 se ha decidido realizar las siguientes tareas:

\begin{itemize}
\item Establecer el patrón de diseño fachada en la aplicación.
\item Decidir el modelo clasificador que ejecuta la aplicación.
\item Hacer compatible la aplicación para \textit{Ubuntu}.
\end{itemize}

\subsection{Sprint12 - 22 Diciembre 2022 - 13 Enero 2023}

El Sprint 12 es un periodo largo debido a que comprende los días de Navidad. En este periodo se ha querido seguir con el enfoque en memoria y anexos para llegar a una versión cercana a la final. Además se ha creado un vídeo manual sobre la aplicación para incluir en el proyecto.

\begin{itemize}
\item Memoria
\item Anexos
\item Vídeo manual de uso aplicación.
\item Versión final de la aplicación \textit{RadarWave}
\item Reestructurar repositorio.
\end{itemize}

\subsection{Sprint13 - 14 Enero 2023 - 19 Enero 2023}

El Sprint 13 comprende el periodo final del desarrollo del proyecto.

Se han realizado la últimas tareas para la entrega del proyecto:

\begin{itemize}
\item Documentación declaración de entrega.
\item Preparar tres ejemplares del TFG en soporte informático compatible.
\item Copia impresa de la memoria del TFG.
\item Dos copias en UBUVirtual de memoria y anexos del TFG.
\item Presentación para el día de la defensa.
\item Entrega del proyecto.
\end{itemize}


\section{Estudio de viabilidad}

\subsection{Viabilidad económica}
En este apartado se explicarán los aspectos económicos del proyecto. Se va a tener en cuenta el coste de personal, el software empleado y los equipos utilizados.
\subsubsection{Costes de personal}
En el proyecto solo ha participado una persona por aproximadamente 6
meses. Se considera un salario de:


\tablaSmallSinColores{Costes de personal}{p{6.4cm} p{2.15cm} p{8cm}}{costespersonales}{
  \multicolumn{1}{p{4.5cm}}{\textbf{Concepto}} & \textbf{Coste{}}\\
 }
 {
  Salario mensual neto  & \multicolumn{1}{r}{1224.75€}\\
  Retención IRPF (12\%) & \multicolumn{1}{r}{180.00€}\\
  Seguridad social (Empleado) (6,35\%) & \multicolumn{1}{r}{95.25€}\\
  Salario mensual bruto  & \multicolumn{1}{r}{1500.00€}\\\hline
  Salario total (6 meses)  & \multicolumn{1}{r}{8500.00€}\\\hline
  Seguridad social (Empresa) (29,9\%) & \multicolumn{1}{r}{448.50€}\\\hline
  Coste total mensual de la empresa & \multicolumn{1}{r}{1794.00€}\\\hline
  }
  
\subsubsection{Costes de software}

A nivel de software no tenemos gastos ya que las librerías son gratuitas y no necesitamos pagar ninguna licencia para el uso del radar.

\subsubsection{Costes de hardware}

En este apartado se describen los costes relacionados con el equipamiento hardware que se ha utilizado para el desarrollo del proyecto. Para calcular el coste amortizado, se ha tenido en cuenta que el tiempo de uso coincide con la duración del proyecto (6 meses) y que su vida útil gira en torno a los 5 años.


\begin{table}[H]
	\centering
	\begin{tabular}{l|l|l}
		\toprule
		Concepto & Coste & Coste amortizado  \\
		\midrule
		Raspberry Pi 4  & 88.40€ & 8.84€ \\
  		Radar A111 & 52.73€ & 5.27€\\
  		Placa XR112 & 176.78€ & 17.67€\\
  		Cable flexible para XR112  & 7.99€ & 0.79€\\
  		Tarjeta SD  & 4.90€ & 0.49€\\
  		Teclado USB & 9.90€ & 0.99€\\
  		Ratón USB & 7.64€ & 0.76€\\
  		Monitor con HDMI & 79.99€ & 7.99€\\
  		Cable HDMI & 7.95€  & 0.79€\\
		\bottomrule
		Coste TOTAL & 444.23€ & 44.42€\\
	\end{tabular}
	\caption{Coste del Hardware}
	\label{tab:material}
\end{table}
 
\subsection{Viabilidad legal}

Uno de los factores más importantes a tener en cuenta en el desarrollo de un proyecto es escoger el tipo de licencia con el que se distribuirá cada una de sus partes. De esta forma, se define el marco legal en el que se puede utilizar cada parte, es decir, lo que se autoriza a hacer y lo que no.

A la hora de determinar si una biblioteca es válida para nuestro proyecto, se tiene en cuenta si la licencia es compatible con \textit{MIT}. La licencia \textit{MIT} permite que el software sea redistribuido libremente. Se ha empleado esta licencia porque la aplicación donde se incluye este proyecto tiene esta licencia. Al igual que todas las demás licencias que hemos utilizado como \textit{BSD} o \textit{Apache 2.0 opensource} que son libres para su redistribución.

\tablaSmallSinColores{Licencias}{p{6.4cm} p{2.15cm} p{8cm}}{licencias}{
  \multicolumn{1}{p{4.5cm}}{\textbf{Librería}} & \textbf{Licencia}\\
 }
 {
  Tensorflow & \multicolumn{1}{r}{Apache 2.0 opensource license}\\
  NumPy & \multicolumn{1}{r}{Nueva Licencia BSD}\\
  Pandas & \multicolumn{1}{r}{Licencia BSD}\\
  Scikit-learn & \multicolumn{1}{r}{Licencia BSD}\\
  Auto-sklearn & \multicolumn{1}{r}{Licencia BSD}\\
  Tcl/Tk & \multicolumn{1}{r}{Licencia BSD}\\
  Socket & \multicolumn{1}{r}{Licencia MIT}\\
  Paramiko & \multicolumn{1}{r}{Licencia LGPL}\\
  matplotlib & \multicolumn{1}{r}{Licencia BSD}\\
  joblib & \multicolumn{1}{r}{Licencia BSD}\\
  TabPFN & \multicolumn{1}{r}{Licencia BSD}\\
  }