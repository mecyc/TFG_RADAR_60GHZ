\apendice{Plan de Proyecto Software}

\section{Introducción}
En este apartado se va a detallar la planificación temporal del proyecto,
especificando las tareas. También se realizarán estimaciones económicas y la
viabilidad legal.
\section{Planificación temporal}

Se ha decidido seguir la metodología SCRUM para la gestión del proyecto.
Se han aplicado las siguientes características:
El desarrollo se ha dividido en iteraciones (sprints), cada una de las cuales ha tenido una duración entre 1 y 2 semanas. Cada iteración se compone de una serie de tareas cuyo principal cometido es incluir nuevas funcionalidades a la infraestructura.
Las tareas de cada sprint son elaboradas en la reunión que se establece al terminar la iteración anterior. Además de planifcar las tareas de la siguiente se supervisan las tareas anteriores.

A continuación, se describen brevemente cada uno de los sprints ejecutados durante el desarrollo del proyecto.

\subsection{Sprint1 - 29 Octubre 2021 - 12 Noviembre 2021}
Sprint que da comienzo al TFG en la modalidad ONLINE. La realización del TFG comenzó en abril de 2021 para la modalidad presencial, actualmente se ha cambiado a la modalidad online solicitando una ampliación para continuar realizando el TFG sobre el mismo tema. Por esta razón hay poca actividad en el repositorio hasta el día de hoy.

Este Sprint comprende las tareas de:
\begin{itemize}
\item Familiarización con GitHub (milestone, commits, estadisticas...)
\item Añadir Readme al repositorio.
\item Desarrollo de la memoria del TFG
\end{itemize}

\subsection{Sprint2}
En el Sprint2 se seguirá con la documentación de la memoria y se barajará crear una interfaz alternativa a la actuar con alguna herramienta drag and drop para código python.

Este Sprint comprende las tareas de:

\begin{itemize}
\item Conceptos teóricos memoria.
\item Técnicas y herramientas memoria.
\item Familiarización con los anexos de la memoria.
\item Crear interfaz alternativa.
\end{itemize}

\subsection{Sprint3}

\subsection{Sprint4}

\subsection{Sprint5}

\section{Estudio de viabilidad}

\subsection{Viabilidad económica}
En este apartado se explicarán los aspectos económicos del proyecto. Se va a tener en cuenta el coste de personal, el software empleado y los equipos utilizados.
\subsubsection{Costes de personal}
En el proyecto solo ha participado una persona por aproximadamente 6
meses. Se considera un salario de:


\tablaSmallSinColores{Costes de personal}{p{6.4cm} p{2.15cm} p{8cm}}{costespersonales}{
  \multicolumn{1}{p{4.5cm}}{\textbf{Concepto}} & \textbf{Coste{}}\\
 }
 {
  Salario mensual neto  & \multicolumn{1}{r}{1.224,75€}\\
  Retención IRPF (12\%) & \multicolumn{1}{r}{180,00€}\\
  Seguridad social (Empleado) (6,35\%) & \multicolumn{1}{r}{95,25€}\\
  Salario mensual bruto  & \multicolumn{1}{r}{1.500,00€}\\\hline
  Salario total (6 meses)  & \multicolumn{1}{r}{8.500,00€}\\\hline
  Seguridad social (Empresa) (29,9\%) & \multicolumn{1}{r}{448,50€}\\\hline
  Coste total mensual de la empresa & \multicolumn{1}{r}{1.794,00€}\\\hline
  }
  
\subsubsection{Costes de software}

A nivel de software no tenemos gastos ya que las librerías son gratuitas y no necesitamos pagar ninguna licencia para el uso del radar.

\subsubsection{Costes de hardware}

En este apartado se describen los costes relacionados con el equipamiento hardware que se ha utilizado para el desarrollo del proyecto. Para calcular el coste amortizado, se ha tenido en cuenta que el tiempo de uso coincide con la duración del proyecto (6 meses) y que su vida útil gira en torno a los 5 años.


\begin{table}[H]
	\centering
	\begin{tabular}{l|l|l}
		\toprule
		Concepto & Coste & Coste amortizado  \\
		\midrule
		Raspberry Pi 4  & 88,40€ & 8,84€ \\
  		Radar A111 & 52,73€ & 5,27€\\
  		Placa XR112 & 176,78€ & 17,67€\\
  		Cable flexible para XR112  & 7,99€ & 0,79€\\
  		Tarjeta SD  & 4,90€ & 0,49€\\
  		Teclado USB & 9,90€ & 0,99€\\
  		Ratón USB & 7,64€ & 0,76€\\
  		Monitor con HDMI & 79,99€ & 7,99€\\
  		Cable HDMI & 7,95€  & 0,79€\\
		\bottomrule
		Coste TOTAL & 444,23€ & 44,42€\\
	\end{tabular}
	\caption{Coste del Hardware}
	\label{tab:material}
\end{table}
 
\subsection{Viabilidad legal}

Uno de los factores más importante a tener en cuenta en el desarrollo de un proyecto es escoger el tipo de licencia con el que se distribuirá cada una de sus partes. De esta forma, se define el marco legal en el que se puede utilizar cada parte, es decir, lo que se autoriza a hacer y lo que no.

A la hora de determinar si una biblioteca es válida para nuestro proyecto, se tiene en cuenta si la licencia es compatible con MIT. La licencia MIT permite que el software sea redistribuido libremente, se ha empleado esta licencia porque la aplicación donde se incluye este proyecto tiene esta licencia.

\tablaSmallSinColores{Licencias}{p{6.4cm} p{2.15cm} p{8cm}}{licencias}{
  \multicolumn{1}{p{4.5cm}}{\textbf{Librería}} & \textbf{Licencia}\\
 }
 {
  Tensorflow & \multicolumn{1}{r}{Apache 2.0 opensource license}\\
  NumPy & \multicolumn{1}{r}{Nueva Licencia BSD}\\
  Pandas & \multicolumn{1}{r}{Licencia BSD}\\
  Scikit-learn & \multicolumn{1}{r}{Licencia BSD}\\
  Tcl/Tk & \multicolumn{1}{r}{Licencia BSD}\\
  }