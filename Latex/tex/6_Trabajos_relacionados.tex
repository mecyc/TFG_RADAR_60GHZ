\capitulo{6}{Trabajos relacionados}

En este apartado se muestran las herramientas relacionadas con el proyecto.
Hoy en día hay poca información y documentación de este tipo de tecnología de radares en la red, por ello solo muestro una única herramienta.

\section{RadarCat}
RadarCat (Radar Categorization for Input and Interaction) permite que un dispositivo electrónico pueda reconocer y clasificar distintos materiales y objetos a tiempo real con alta precisión. Además da mucha más información de los objetos, como su estructura interna.

El escaner/radar utilizado envía ondas electromagnéticas al objeto, estas rebotan y vuelven al punto de partida para su procesamiento con aprendizaje automático o machine learning.

La figura 6.1 muestra un esquema de cómo está estructurado RadarCar. Podemos ver que es necesario un ordenador dónde está instalado el software de RadarCat, a este equipo se conecta una placa base donde están las antenas capaces de realizar las lecturas de los objetos.
Por último encontramos una placa que protege el sensor y soporta el objeto a reconocer.

\imagen{radarcat}{RadarCat.}

Más allá de la interacción humana con la computadora, RadarCat también abre nuevas oportunidades en áreas como la navegación y el conocimiento mundial (por ejemplo, usuarios con baja visión).