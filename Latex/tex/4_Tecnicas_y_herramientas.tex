\capitulo{4}{Técnicas y herramientas}

\section{Técnicas y metodologías}

\subsection{Metodología SCRUM}
Se trata de una metodología de trabajo ágil que tiene como finalidad dividir en periodos de tiempo el flujo de trabajo. Estos periodos son conocidos como  \textit{sprints}, al finalizar cada \textit{sprint} se realizan revisiones y reuniones donde se deciden las tareas de los próximos \textit{sprints}.

\subsection{Cliente de control de versiones}
\begin{itemize}
\item Herramientas consideradas: GitHub Desktop y Gitkraken
\item Herramienta elegida: GitHub Desktop
\end{itemize}

GitHub Desktop utilizado para la gestión ágil del proyecto.

\subsection{Hosting del repositorio}
\begin{itemize}
\item Herramientas consideradas: GitLab y GitHub.
\item Herramienta elegida: GitHub
\end{itemize}

GitHub ofrece una gran cantidad de facilidades para mantener el proyecto en la nube y debido a que no cobra por sus servicios lo convierte en la mejor opción posible. Nos permite alojar nuestro repositorio central del proyecto usando el control de versiones Git.

Git es un sistema de control de versiones distribuido de código abierto.
El control de versiones nos permitirá retornar a algún punto anterior del desarrollo de nuestra aplicación en caso de sufrir errores.

Url de la herramienta: \url{https://github.com/}


\section{Lenguajes y bibliotecas}

\subsection{Python}
Python es un lenguaje de programación que destaca por su código limpio y legible esto hace que sea uno de los lenguajes de iniciación de muchos programadores.
Además se trata de un lenguaje de multiparadigma y multiplataforma muy utilizado en la técnica del BigData.

Url de la herramienta: \url{https://www.python.org/}

\subsection{TensorFlow}
TensorFlow es una biblioteca de software de código abierto para computación numérica. Desarrollada por Google capaz de construir y entrenar redes neuronales.

Tensorflow es capaz de realizar cálculos
numéricos usando grafos de flujo de datos. Los nodos hacen referencia a las operaciones matemáticas y los enlaces representan conjuntos de datos multidimensionales que relacionan los nodos.

La arquitectura flexible de Tensorflow permite realizar los cálculos en una o varias CPUs o GPUs mediante una sola API.

Url de la herramienta: \url{https://www.tensorflow.org/}

\subsection{NumPy}
NumPy es una biblioteca utilizada en la programacion con Python para crear vectores y matrices grandes multidimensionales.

La funcionalidad principal de NumPy es su estructura de datos "ndarray", para una matriz de n dimensiones.

El uso de NumPy en Python brinda una funcionalidad comparable a MATLAB \cite{Matlab}.

Url de la herramienta: \url{https://numpy.org/}

\subsection{Pandas}
Pandas es una biblioteca de código abierto especializada en el manejo y análisis de estructuras de datos flexible y fácil de usar, construida sobre el lenguaje de programación Python.

Su objetivo es ser el bloque de construcción fundamental de alto nivel para realizar análisis de datos prácticos del mundo real en Python.

Pandas está construido sobre NumPy y está destinado a integrarse bien dentro de un entorno informático científico con muchas otras bibliotecas de terceros.

Url de la herramienta: \url{https://pandas.pydata.org/}

\subsection{acconeer.exptool}
acconeer.exptool es la biblioteca facilitada por la empresa Acconeer. Con ella pondremos en funcionamiento el radar de 60GHz además de recopilar los datos necesarios.

Url de la herramienta: \url{https://github.com/acconeer/acconeer-python-exploration}

\subsection{Sklearn}
Scikit-learn es una biblioteca para aprendizaje automático de software libre para el lenguaje de programación de Python.
En está biblioteca encontramos varios algoritmos de clasificación, regresión y análisis de grupos entre los cuales están máquinas de vectores de soporte, bosques aleatorios, Gradient boosting, K-means y el utilizado en este proyecto RandomForest.

Url de la herramienta: \url{https://scikit-learn.org/}

\subsection{Tkinter}
Tkinter es una adaptación de la biblioteca gráfica Tcl/Tk \cite{Tcl} para el lenguaje de programación Python. Mediante Tkinter se desarrollará la interfaz gráfica para comparar distintos tipos de materiales.

Url de la herramienta: \url{https://docs.python.org/es/3/library/tkinter.html}

\section{Herramientas de desarrollo}
\subsection{Jupyter Notebook}
Jupyter Notebook es el entorno de trabajo utilizado en el proyecto que permite desarrollar código en Python de manera dinámica. Nos ofrece integrar en un mismo archivo bloques de código, texto, gráficas o imágenes. Utilizado ampliamente en análisis numéricos y estadísticos.

Jupyter admite más de 40 lenguajes de programación, incluidos Python, R, Julia y Scala.

Url de la herramienta: \url{https://jupyter.org/}

\section{Herramientas de documentación}
\subsection{\LaTeX}
\LaTeX es un sistema de elaboración de documentos basado en comandos. El proyecto está desarrollado mediante una plantilla creada en \LaTeX.

Url de la herramienta: \url{https://jupyter.org/}

\subsection{MiKTeX}
MiKTeX es una distribución de \LaTeX encargada de gestionar los componentes y paquetes. Tiene la capacidad de actualizarse así mismo descargando nuevas versiones de componentes.

Url de la herramienta: \url{https://miktex.org/}

\subsection{Texmaker}
Texmaker es el editor de documentos \LaTeX utilizado para funcionar necesita MiKTeX.

Url de la herramienta: \url{https://www.xm1math.net/texmaker/}

\subsection{BibItNow!}
BibItNow! es la extensión del navegador que utilizamos para citar páginas web en el formato Bibtex, formato soportado por \LaTeX.

Url de la herramienta: \url{https://chrome.google.com/webstore/detail/bibitnow/bmnfikjlonhkoojjfddnlbinkkapmldg}
