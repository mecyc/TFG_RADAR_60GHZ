\capitulo{4}{Técnicas y herramientas}

\section{Técnicas y metodologías}

\subsection{Metodología SCRUM}
Se trata de una metodología de trabajo ágil que tiene como finalidad dividir en periodos de tiempo el flujo de trabajo. Estos periodos son conocidos como  \textit{sprints}, al finalizar cada \textit{sprint} se realizan revisiones y reuniones donde se deciden las tareas de los próximos \textit{sprints}.

\subsection{Cliente de control de versiones}
\begin{itemize}
\item Herramientas consideradas: \textit{GitHub Desktop} y \textit{Gitkraken}
\item Herramienta elegida: \textit{GitHub Desktop}
\end{itemize}

\textit{GitHub Desktop} utilizado para la gestión ágil del proyecto.

\subsection{Hosting del repositorio}
\begin{itemize}
\item Herramientas consideradas: \textit{GitLab} y \textit{GitHub}
\item Herramienta elegida: \textit{GitHub}
\end{itemize}

\textit{GitHub} ofrece una gran cantidad de facilidades para mantener el proyecto en la nube y debido a que no cobra por sus servicios lo convierte en la mejor opción posible. Nos permite alojar nuestro repositorio central del proyecto usando el control de versiones \textit{Git}.

\textit{Git} es un sistema de control de versiones distribuido de código abierto. El control de versiones nos permitirá retornar a algún punto anterior del desarrollo de nuestra aplicación en caso de sufrir errores.


Url de la herramienta: \url{https://github.com/}


\section{Lenguajes y bibliotecas}

\subsection{\textit{Python}}

El lenguaje de programación \textit{Python} se diferencia por su código el cual es legible y limpio esto hace que sea uno de los lenguajes de iniciación de muchos programadores. Además, se trata de un lenguaje de multiparadigma y multiplataforma muy utilizado en la técnica del \textit{BigData}.

Url de la herramienta: \url{https://www.python.org/}

\subsection{\textit{NumPy}}

\textit{NumPy} es una biblioteca utilizada en la programacion con \textit{Python} para crear vectores y matrices grandes multidimensionales.

La funcionalidad principal de \textit{NumPy} es su estructura de datos \textit{«ndarray»}, para una matriz de \textit{n} dimensiones.

El uso de \textit{NumPy} en \textit{Python} brinda una funcionalidad comparable a \textit{MATLAB}.

Url de la herramienta: \url{https://numpy.org/}

\subsection{\textit{Pandas}}

\textit{Pandas} es una biblioteca de código abierto especializada en el manejo y análisis de estructuras de datos flexible y fácil de usar, construida sobre el lenguaje de programación \textit{Python}.

Su objetivo es ser el bloque de construcción fundamental de alto nivel para realizar análisis de datos prácticos del mundo real en \textit{Python}.

\textit{Pandas} está construido sobre \textit{NumPy} y está destinado a integrarse bien dentro de un entorno informático científico con muchas otras bibliotecas de terceros.

Url de la herramienta: \url{https://pandas.pydata.org/}

\subsection{\textit{acconeer.exptool}}

\textit{acconeer.exptool} es la biblioteca facilitada por la empresa \textit{Acconeer}. Con ella pondremos en funcionamiento el radar de 60GHz además de recopilar los datos necesarios.

Url de la herramienta: \url{https://github.com/acconeer/acconeer-python-exploration}


\subsection{\textit{Sklearn}}

\textit{Scikit-learn} es una biblioteca para aprendizaje automático de software libre para el lenguaje de programación de \textit{Python}.
En esta biblioteca encontramos varios algoritmos de clasificación, regresión y análisis de grupos entre los cuales están máquinas de vectores de soporte, bosques aleatorios, \textit{Gradient boosting}, \textit{K-means}...

Url de la herramienta: \url{https://scikit-learn.org/}

\subsection{\textit{Auto-sklearn}}

\textit{Auto-sklearn} es un kit de herramientas de aprendizaje automático automatizado de código libre y creado a partir de \textit{Scikit-learn}.

En esta biblioteca encontramos un total de 15 algoritmos de clasificación, 14 algoritmos de preprocesamiento de funciones además se ocupa del escalado de datos, la codificación de parámetros categóricos y los valores faltantes.

Url de la herramienta: \url{https://automl.github.io/auto-sklearn/master/}

\subsection{\textit{TabPFN}}

\textit{TabPFN} método clasificador de software libre que puede realizar una clasificación supervisada para pequeños conjuntos de datos en menos de un segundo.

Url de la herramienta: \url{https://www.automl.org/tabpfn-a-transformer-that-solves-small-tabular-classification-problems-in-a-second/}

\subsection{\textit{Tkinter}}

\textit{Tkinter} es una adaptación de la biblioteca gráfica \textit{Tcl/Tk} \cite{Tcl} para el lenguaje de programación \textit{Python}. Mediante \textit{Tkinter} se desarrollará la interfaz gráfica para comparar distintos tipos de materiales.

Url de la herramienta: \url{https://docs.python.org/es/3/library/tkinter.html}

\subsection{\textit{cmath}}
Módulo incorporado en \textit{Python} que usaremos para tareas matemáticas. Las funciones de este módulo aceptan números enteros, números de coma flotante o números complejos como argumentos.

Url de la herramienta: \url{https://docs.python.org/3/library/cmath.html}

\subsection{\textit{Socket}}
Es una librería que proporciona acceso a la interfaz \textit{BSD}\footnote{Distribución de software Berkeley.} \textit{socket}. En el presente proyecto se utilizará para realizar las comunicaciones con el radar.

Url de la herramienta: \url{https://docs.python.org/es/3.10/library/socket.html}

\subsection{\textit{Paramiko}}
\textit{Paramiko} es un modulo de \textit{Python} utilizado para control remoto. En este proyecto nos ayudaremos de esta biblioteca para ejecutar comandos en el radar.

Url de la herramienta: \url{https://www.paramiko.org/}

\subsection{\textit{matplotlib}}
\textit{Matplotlib} es una biblioteca para la generación de gráficos a partir de datos almacenados en listas o arrays. Se empleará para visualizar gráficamente los resultados de la clasificación.

Url de la herramienta: \url{https://matplotlib.org/}

\subsection{\textit{joblib}}

\textit{Joblib} es un conjunto de herramientas encargadas de realizar una canalización ligera en \textit{Python} mejorando la rapidez del tratamiento de los datos.

Url de la herramienta: \url{https://docs.python.org/3/library/cmath.html}

\section{Herramientas de desarrollo}
\subsection{\textit{Jupyter Notebook}}
\textit{Jupyter Notebook} es el entorno de trabajo utilizado en el proyecto que permite desarrollar código en \textit{Python} de manera dinámica. Nos ofrece integrar en un mismo archivo bloques de código, texto, gráficas o imágenes. Utilizado ampliamente en análisis numéricos y estadísticos.

\textit{Jupyter} admite más de 40 lenguajes de programación, incluidos \textit{Python, R, Julia} y \textit{Scala}.

Url de la herramienta: \url{https://jupyter.org/}

\subsection{\textit{Colab}}
\textit{Colab} o \textit{Colaboratory} es un producto de \textit{Google Research}. Permite a cualquier usuario escribir y ejecutar código arbitrario de \textit{Python} en el navegador, es una alternativa a \textit{Jupyter Notebook}.

Se ha utilizado para ejecutar \textit{auto-sklearn} y \textit{TabPFN}. Debido a que para utilizar \textit{auto-sklearn} se necesita un crea un entorno en \textit{Ubuntu} y \textit{TabPFN} estaba dando problemas de instalación en el equipo utilizado.

Url de la herramienta: \url{https://colab.research.google.com/}

\section{Herramientas de documentación}
\subsection{\LaTeX}
\LaTeX{} es un sistema de elaboración de documentos basado en comandos. El proyecto está desarrollado mediante una plantilla creada en \LaTeX.

Url de la herramienta: \url{https://jupyter.org/}

\subsection{\textit{MiKTeX}}
\textit{MiKTeX} es una distribución de \LaTeX{} encargada de gestionar los componentes y paquetes. Tiene la capacidad de actualizarse así mismo descargando nuevas versiones de componentes.

Url de la herramienta: \url{https://miktex.org/}

\subsection{\textit{Texmaker}}
\textit{Texmaker} es el editor de documentos \LaTeX{} utilizado para funcionar necesita \textit{MiKTeX}.

Url de la herramienta: \url{https://www.xm1math.net/texmaker/}

\subsection{\textit{BibItNow!}}
\textit{BibItNow!} es la extensión del navegador que utilizamos para citar páginas web en el formato \textit{Bibtex}, formato soportado por \LaTeX.

Url de la herramienta: \url{https://chrome.google.com/webstore/detail/bibitnow/bmnfikjlonhkoojjfddnlbinkkapmldg}
