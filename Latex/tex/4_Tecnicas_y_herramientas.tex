\capitulo{4}{Técnicas y herramientas}

\section{Técnicas y metodologías}

\subsection{Metodología SCRUM}
Se trata de una metodología de trabajo ágil que tiene como finalidad dividir en periodos de tiempo el flujo de trabajo. Estos periodos son conocidos como  \textit{sprints}, al finalizar cada \textit{sprint} se realizan revisiones y reuniones donde se deciden las tareas de los próximos \textit{sprints}.

\subsection{Cliente de control de versiones}
\begin{itemize}
\item Herramientas consideradas: GitHub Desktop y Gitkraken
\item Herramienta elegida: GitHub
\end{itemize}

GitHub basado en Git junto con la extensión ZenHub se han utilizado para la gestión ágil del proyecto.

\subsection{Hosting del repositorio}
\begin{itemize}
\item Herramientas consideradas: GitLab y GitHub.
\item Herramienta elegida: GitHub
\end{itemize}

GitHub ofrece una gran cantidad de facilidades para mantener el proyecto en la nube y debido a que no cobra por sus servicios lo convierte en la mejor opción posible.


\section{Lenguajes y bibliotecas}

\subsection{Python}
Python es un lenguaje de programación que destaca por su código limpio y legible esto hace que sea uno de los lenguajes de iniciación de muchos programadores.
Además se trata de un lenguaje de multiparadigma y multiplataforma muy utilizado en la técnica del BigData.

\subsection{TensorFlow}
TensorFlow es una biblioteca de software de código abierto para computación numérica. Desarrollada por Google capaz de construir y entrenar redes neuronales.

\subsection{NumPy}
NumPy es una biblioteca utilizada en la programacion con Python para crear vectores y matrices grandes multidimensionales.

\subsection{Pandas}
Pandas es una biblioteca especializada en el manejo y análisis de estructuras de datos.

\subsection{acconeer.exptool}
acconeer.exptool es la biblioteca facilitada por la empresa Acconeer. Con ella pondremos en funcionamiento el radar de 60GHz además de recopilar los datos necesarios.

\subsection{Sklearn}
Tkinter es una adaptación de la biblioteca gráfica Tcl/Tk para el lenguaje de programación Python. Mediante Tkinter se desarrollará la interfaz gráfica para comparar distintos tipos de materiales.

\subsection{Tkinter}
Tkinter es una adaptación de la biblioteca gráfica Tcl/Tk\cite{Tcl} para el lenguaje de programación Python. Mediante Tkinter se desarrollará la interfaz gráfica para comparar distintos tipos de materiales.

\section{Herramientas de desarrollo}
\subsection{Jupyter Notebook}
Jupyter Notebook es el entorno de trabajo utilizado en el proyecto que permite desarrollar código en Python de manera dinámica. Nos ofrece integrar en un mismo archivo bloques de código, texto, gráficas o imágenes. Utilizado ampliamente en análisis numéricos y estadísticos.

\section{Herramientas de documentación}
\subsection{\LaTeX}
\LaTeX es un sistema de elaboración de documentos basado en comandos. El proyecto está desarrollado mediante una plantilla creada en \LaTeX.

\subsection{MiKTeX}
MiKTeX es una distribución de \LaTeX encargada de gestionar los componentes y paquetes. Tiene la capacidad de actualizarse así mismo descargando nuevas versiones de componentes.

\subsection{Texmaker}
Texmaker es el editor de documentos \LaTeX utilizado para funcionar necesita MiKTeX.

\subsection{BibItNow!}
BibItNow! es la extensión del navegador que utilizamos para citar páginas web en el formato Bibtex, formato soportado por \LaTeX.
