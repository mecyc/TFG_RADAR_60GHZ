\apendice{Documentación de instalación}

\section{Introducción}

En este apartado se explica el proceso de instalación y ejecución de la aplicación desarrollada para el proyecto. Se incluye el manual de instalación tanto para \textit{Windows} como para {Linux}.

\section{Instalación y ejecución en \textit{Windows}} \label{sec:windows}

La siguiente instalación se ha realizado en un equipo con las siguientes características:
\begin{itemize}
\item[•] \textbf{SISTEMA OPERATIVO:} \textit{Microsoft Windows 10 Pro}
\item[•] \textbf{PROCESADOR:} \textit{Intel(R) Core(TM) i7-4720HQ CPU @ 2.60GHz}
\item[•] \textbf{MEMORIA RAM:} 12 GB
\item[•] \textbf{TARJETA GRÁFICA:} \textit{NVIDIA GeForce 930M}
\end{itemize}

En realidad la aplicación \textit{RadarWave} no hace falta que se instale, pero si necesitamos instalar las librerías utilizadas por el programa.

Antes de proceder con la instalación de las librerías vamos a verificar si tenemos instalado Python, de no ser así procedemos a su instalación.
\begin{itemize}
\item Descargamos el instalador desde la web de \href{https://www.python.org/downloads/}{\textit{Python}}.
\item Ejecutamos el archivo descargado.
\item En la pantalla de de inicio de la instalación marcamos la opción de añadir \textit{Python} al \textit{PATH}, si no luego se tendrá que hacer manualmente.
\item Pulsamos en instalar ahora (\textit{install now})

\imagen{path}{Instalación de \textit{Python}.}

\end{itemize}

Una vez instalado \textit{Python} lo primero que haremos será descargar y descomprimir el fichero \textit{RadarWave\_v1.0.zip}, se puede descargar desde \url{https://github.com/mecyc/TFG_RADAR_60GHZ/blob/main/RadarWave_v1.0.zip} 

Procedemos a instalar las librerías, están incluidas en el archivo \textit{requirements.txt}. Primero instalamos \textit{setuptools} que facilita el empaquetado de proyectos de \textit{Python} y a continuación el resto, para ello debemos abrir la consola y lanzar las siguientes instrucciones.

\begin{verbatim}
python -m pip install -U --user setuptools wheel
python -m pip install -U --user -r requirements.txt
\end{verbatim}

\imagen{requirements}{Instalar librerías.}

Finalmente se debe instalar la librería de \textit{Acconeer}, para ello descargamos del repositorio del TFG la carpeta \href{https://github.com/mecyc/TFG_RADAR_60GHZ/tree/main/acconeer-python-exploration}{\textit{acconer-python-exploration}}. Una vez descargada abrimos el terminal dentro de la carpeta y ejecutamos la siguiente orden.\cite{Acconeer2021}

\begin{verbatim}
python -m pip install -U --user .
\end{verbatim}

\imagen{instalar_libacconer}{Instalar librería \textit{Acconeer}.}

Una vez instaladas las librerías procedemos a ejecutar el programa haciendo doble clic en el fichero \textit{RadarWave.py} donde está el programa.

Programa ejecutado y probado en \textit{Windows}:

\imagen{radarwaveWindows}{\textit{RadarWave} en \textit{Windows}.}

\section{Instalación y ejecución en \textit{Linux}} \label{sec:linux}

La siguiente instalación se ha realizado en un equipo virtualizado en \textit{Oracle VM VirtualBox 6.1} con las siguientes características:
\begin{itemize}
\item[•] \textbf{SISTEMA OPERATIVO:} \textit{Ubuntu 22.04 LTS}
\item[•] \textbf{PROCESADOR:} \textit{Intel(R) Core(TM) i7-4720HQ CPU @ 2.60GHz}
\item[•] \textbf{MEMORIA RAM:} 4 GB
\end{itemize}

En realidad la aplicación \textit{RadarWave} no hace falta que se instale, pero si necesitamos instalar las librerías utilizadas por el programa.

El sistema operativo que utilizamos es \textit{Ubuntu}, una distribución {Linux}, por lo que \textit{Python} está integrado por defecto y no hace falta su instalación. Se recomienda tener actualizado \textit{Python} a la última versión.

Antes de instalar las librerías necesarias instalamos \textit{pip}, es un sistema de gestión de paquetes que nos ayuda a instalar las librerías, y \textit{setuptools}, facilita el empaquetado de proyectos de Python.

\begin{verbatim}
sudo apt install python3-pip
pip3 install setuptools-rust
\end{verbatim}

Una vez terminada la instalación procedemos a descargar y descomprimir el fichero \textit{RadarWave\_v1.0.zip}, se puede descargar desde \url{https://github.com/mecyc/TFG_RADAR_60GHZ/blob/main/RadarWave_v1.0.zip}

A continuación se debe instalar las librerías que usa la interfaz desarrollada. Indicar que para ejecutar el siguiente comando nos debemos situar en la dirección o \textit{PATH} donde se encuentra el archivo \textit{requirements.txt}, lo encontramos dentro del fichero que acabamos de descomprimir.

El comando sería el siguiente:
\begin{verbatim}
pip3 install -r requirements.txt
\end{verbatim}

Si existiera algún problema al instalar la librería de \textit{Tkinter} (tk) se deben lanzar los siguientes comandos:
\begin{verbatim}
sudo apt-get install python3-tk
sudo apt-get install python3-pil python3-pil.imagetk
\end{verbatim}

Finalmente se debe instalar la librería de \textit{Acconeer}, para ello descargamos del repositorio del TFG la carpeta \href{https://github.com/mecyc/TFG_RADAR_60GHZ/tree/main/acconeer-python-exploration}{\textit{acconer-python-exploration}}. Una vez descargada abrimos el terminal dentro de la carpeta y ejecutamos la siguiente orden.

\begin{verbatim}
python3 setup.py install
\end{verbatim}

Si queremos ejecutar \textit{RadarWave} en \textit{Ubuntu} deber ser lanzando el siguiente comando en la misma ruta del archivo \textit{RadarWave.py}.

\begin{verbatim}
python3 RadarWave.py
\end{verbatim}


Programa ejecutado y probado en \textit{Ubuntu}:

\imagen{radarwaveUbuntu}{\textit{RadarWave} en \textit{Ubuntu}.}


