\apendice{Especificación de Requisitos}

\section{Introducción}

En este apéndice se especifican las necesidades del cliente o de los usuarios, los requisitos y la especificación de los mismos. Se analizan tanto los requisitos funcionales como los no funcionales.

\section{Objetivos generales}

Los objetivos generales del proyecto se pueden describir en los siguientes puntos:
\begin{itemize}
\item[•] El objetivo principal del proyecto es el desarrollo de una aplicación capaz de comunicarse con el radar y realizar lecturas de objetos para identificar distintos materiales.

\item[•] Documentar los diferentes elementos que componen el ensamblar y poner en funcionamiento un radar de 60GHz fabricado por \textit{Acconeer} el cual es capaz de identificar, mediante un clasificador, el tipo de material al que pertenece un objeto.

\item[•] Crear un procedimiento capaz de extraer determinadas características de las lecturas de los objetos para entrenar un modelo de clasificación automática.

\item[•] Poder generar un gráfico de tasa de acierto, el cual sea capaz de mostrar la clasificación de los diferentes materiales empleados.

\item[•]Se establecerá una interfaz capaz de poder ser implementada en el sistema operativo Windows, buscando que la aplicación pueda ser implementada en cualquier dispositivo que cuente con este sistema.
\end{itemize}

\section{Catalogo de requisitos}
\subsection{Requisitos funcionales}
Los requerimientos funcionales son aquellos que describen cualquier actividad que deba realizar el sistema o \textit{software}. Se relacionan con los casos de uso.

\begin{itemize}
\item \textbf{RF-1 Abrir lecturas}: la aplicación debe permitir al usuario abrir lecturas almacenadas para que sean procesadas.
\item \textbf{RF-2 Reconocimiento de materiales}: la aplicación debe ser capaz de clasificar los datos extraidos de las lecturas.
\item \textbf{RF-3 Conectar con el radar}: la aplicación debe ser capaz de crear una conexión estable entre el equipo y el radar.
\item \textbf{RF-4 Iniciar lectura por radar}: la aplicación debe conseguir leer un objeto desde el radar.
\item \textbf{RF-5 Guardar datos}: la aplicación debe permitir guardar los datos leidos por el radar o abiertos desde el explorador de archivos.
\item \textbf{RF-6 Minimizar la interfaz}: se debe permitir minimizar la interfaz.
\item \textbf{RF-7 Expandir la interfaz}: se debe permitir expandir la interfaz manteniendo el aspecto.
\item \textbf{RF-8 Informe de error}: la aplicación debe informar de un error al usuario no se consiga establecer conexión con el radar.
\end{itemize}

\subsection{Requisitos no funcionales}
Los requisitos no funcionales specifican los criterios a seguir, restricciones y condiciones que impone el cliente.

\begin{itemize}
\item \textbf{RNF-1 Rendimiento}: la aplicación tiene que ser fluida y tener un tiempo de respuesta bajo.
\item \textbf{RNF-2 Usabilidad}: la aplicación debe ser intuitiva y fácil de entender y utilizar.
\item \textbf{RNF-3 Mantenibilidad}: la aplicación debe ser fácilmente modificable.
\item \textbf{RNF-4 Portabilidad}: la aplicación debe poder ejecutarse en distintos equipos con el sistema operativo Windows.
\end{itemize}

\section{Especificación de requisitos}

\subsection{Diagramas de casos de uso}

\imagen{casosdeuso}{Diagrama de casos de uso.}

El actor será la persona que manejará el \textit{software}.

\newpage

\subsection{Casos de uso}

\begin{longtable}[H]{@{}ll@{}}
	\toprule
	\begin{minipage}[b]{0.23\columnwidth}\raggedright\strut
		\textbf{CU-01}\strut
	\end{minipage} & \begin{minipage}[b]{0.71\columnwidth}\raggedright\strut
		\textbf{Abrir fichero.}\strut
	\end{minipage}\tabularnewline
	\midrule
	\endhead  
	\begin{minipage}[t]{0.23\columnwidth}\raggedright\strut
		\textbf{Requisitos asociados}\strut
	\end{minipage} & \begin{minipage}[t]{0.71\columnwidth}\raggedright\strut
		RF-1\strut
	\end{minipage}\tabularnewline
	\begin{minipage}[t]{0.23\columnwidth}\raggedright\strut
		\textbf{Descripción}\strut
	\end{minipage} & \begin{minipage}[t]{0.71\columnwidth}\raggedright\strut
		Permite al usuario abrir ficheros con formato \textit{.npy}, continen lecturas realizadas con anterioridad.\strut
	\end{minipage}\tabularnewline
	\begin{minipage}[t]{0.23\columnwidth}\raggedright\strut
		\textbf{Precondición}\strut
	\end{minipage} & \begin{minipage}[t]{0.71\columnwidth}\raggedright\strut
		El usuario debe tener iniciada la aplicación.\\
		Debe existir un fichero de lecturas anteriores.\strut
	\end{minipage}\tabularnewline
	\begin{minipage}[t]{0.23\columnwidth}\raggedright\strut
		\textbf{Acciones}\strut
	\end{minipage} & \begin{minipage}[t]{0.71\columnwidth}\raggedright\strut
		\begin{enumerate}
			\def\labelenumi{\arabic{enumi}.}
			\tightlist
			\item
			(Opcional) El usuario pulsa en la barra de herramientas en lectura.
			\item
			El usuario hace clic en \textit{Leer archivo} en menu desplegable o icono de documento.
			\item
			Se abre una ventana de explorador de archivos.
			\item
			Se selecciona el fichero a abrir.
		\end{enumerate}\strut
	\end{minipage}\tabularnewline
	\begin{minipage}[t]{0.23\columnwidth}\raggedright\strut
		\textbf{Postcondición}\strut
	\end{minipage} & \begin{minipage}[t]{0.71\columnwidth}\raggedright\strut
		Se activan las posibilidades de \textit{Verificar lectura} y \textit{Guardar lectura}, se iluminan los botones.\strut
	\end{minipage}\tabularnewline
	\begin{minipage}[t]{0.23\columnwidth}\raggedright\strut
		\textbf{Excepciones}\strut
	\end{minipage} & \begin{minipage}[t]{0.71\columnwidth}\raggedright\strut
		Solo se pueden abrir ficheros con la extension \textit{.npy}\strut
	\end{minipage}\tabularnewline
	\begin{minipage}[t]{0.23\columnwidth}\raggedright\strut
		\textbf{Importancia}\strut
	\end{minipage} & \begin{minipage}[t]{0.71\columnwidth}\raggedright\strut
		Alta\strut
	\end{minipage}\tabularnewline
	\bottomrule
	\caption{CU-02 Abrir fichero.}
\end{longtable}

\newpage

\begin{longtable}[H]{@{}ll@{}}
	\toprule
	\begin{minipage}[b]{0.23\columnwidth}\raggedright\strut
		\textbf{CU-02}\strut
	\end{minipage} & \begin{minipage}[b]{0.71\columnwidth}\raggedright\strut
		\textbf{Leer objeto.}\strut
	\end{minipage}\tabularnewline
	\midrule
	\endhead  
	\begin{minipage}[t]{0.23\columnwidth}\raggedright\strut
		\textbf{Requisitos asociados}\strut
	\end{minipage} & \begin{minipage}[t]{0.71\columnwidth}\raggedright\strut
		RF-3, RF-4, RF-8\strut
	\end{minipage}\tabularnewline
	\begin{minipage}[t]{0.23\columnwidth}\raggedright\strut
		\textbf{Descripción}\strut
	\end{minipage} & \begin{minipage}[t]{0.71\columnwidth}\raggedright\strut
		Permite al usuario iniciar la conexión con el radar y obtener los datos de una lectura.\strut
	\end{minipage}\tabularnewline
	\begin{minipage}[t]{0.23\columnwidth}\raggedright\strut
		\textbf{Precondición}\strut
	\end{minipage} & \begin{minipage}[t]{0.71\columnwidth}\raggedright\strut
		El equipo debe estar conectado a la red de internet \\
		El usuario debe tener iniciada la aplicación.\\
		El radar debe estar conectado a la red eléctrica.\\
		El radar debe estar conectado a la misma red de internet que el equipo por \textit{Ethernet} o \textit{Wifi}\strut
	\end{minipage}\tabularnewline
	\begin{minipage}[t]{0.23\columnwidth}\raggedright\strut
		\textbf{Acciones}\strut
	\end{minipage} & \begin{minipage}[t]{0.71\columnwidth}\raggedright\strut
		\begin{enumerate}
			\def\labelenumi{\arabic{enumi}.}
			\tightlist
			\item
			(Opcional) El usuario pulsa en la barra de herramientas en lectura.
			\item
			El usuario hace clic en \textit{Iniciar lectura} o icono de radar.
			\item
			Se inicia la conexión con el radar.
			\item
			Se arranca el servicio de lectura en el radar.
			\item
			Se obtienen los datos de la lectura.
		\end{enumerate}\strut
	\end{minipage}\tabularnewline
	\begin{minipage}[t]{0.23\columnwidth}\raggedright\strut
		\textbf{Postcondición}\strut
	\end{minipage} & \begin{minipage}[t]{0.71\columnwidth}\raggedright\strut
		Se activan las posibilidades de \textit{Verificar lectura} y \textit{Guardar lectura}, se iluminan los botones.\strut
	\end{minipage}\tabularnewline
	\begin{minipage}[t]{0.23\columnwidth}\raggedright\strut
		\textbf{Excepciones}\strut
	\end{minipage} & \begin{minipage}[t]{0.71\columnwidth}\raggedright\strut
		No se consigue realizar la conexión con el radar.\strut
	\end{minipage}\tabularnewline
	\begin{minipage}[t]{0.23\columnwidth}\raggedright\strut
		\textbf{Importancia}\strut
	\end{minipage} & \begin{minipage}[t]{0.71\columnwidth}\raggedright\strut
		Alta\strut
	\end{minipage}\tabularnewline
	\bottomrule
	\caption{CU-02 Leer objeto.}
\end{longtable}

\newpage

\begin{longtable}[H]{@{}ll@{}}
	\toprule
	\begin{minipage}[b]{0.23\columnwidth}\raggedright\strut
		\textbf{CU-03}\strut
	\end{minipage} & \begin{minipage}[b]{0.71\columnwidth}\raggedright\strut
		\textbf{Clasificación de datos.}\strut
	\end{minipage}\tabularnewline
	\midrule
	\endhead  
	\begin{minipage}[t]{0.23\columnwidth}\raggedright\strut
		\textbf{Requisitos asociados}\strut
	\end{minipage} & \begin{minipage}[t]{0.71\columnwidth}\raggedright\strut
		RF-1, RF-2, RF-3, RF-4, RF-8\strut
	\end{minipage}\tabularnewline
	\begin{minipage}[t]{0.23\columnwidth}\raggedright\strut
		\textbf{Descripción}\strut
	\end{minipage} & \begin{minipage}[t]{0.71\columnwidth}\raggedright\strut
		Permite al usuario clasificar los datos extraídos de la lectura por radar o de un archivo.\strut
	\end{minipage}\tabularnewline
	\begin{minipage}[t]{0.23\columnwidth}\raggedright\strut
		\textbf{Precondición}\strut
	\end{minipage} & \begin{minipage}[t]{0.71\columnwidth}\raggedright\strut
		El equipo debe estar conectado a la red de internet \\
		El usuario debe tener iniciada la aplicación.\\
		El radar debe estar conectado a la red eléctrica.\\
		El radar debe estar conectado a la misma red de internet que el equipo por \textit{Ethernet} o \textit{Wifi}\strut
	\end{minipage}\tabularnewline
	\begin{minipage}[t]{0.23\columnwidth}\raggedright\strut
		\textbf{Acciones}\strut
	\end{minipage} & \begin{minipage}[t]{0.71\columnwidth}\raggedright\strut
		\begin{enumerate}
			\def\labelenumi{\arabic{enumi}.}
			\tightlist
			\item
			(Opcional) El usuario pulsa en la barra de herramientas en lectura.
			\item
			A. El usuario hace clic en \textit{Leer archivo} o icono de fichero.\\
			B. El usuario hace clic en \textit{Iniciar lectura} o icono de radar.
			\item
			A. Se inicia la conexión con el radar. \\
			B. Se abre el explorador de archivos.
			\item
			A. Se arranca el servicio de lectura en el radar. \\
			B. El usuario selecciona el archivo deseado.
			\item
			Se obtienen los datos..
			\item
			Se iluminan los botones de \textit{Verificar lectura} y \textit{Gurdar lectura}
			\item 
			El usuario pulsa en \textit{Verificar lectura}
			\item
			Se procesan los datos.
			\item
			La pantalla principal se actualiza mostrando la pertenencia del objeto a un material.
		\end{enumerate}\strut
	\end{minipage}\tabularnewline
	\begin{minipage}[t]{0.23\columnwidth}\raggedright\strut
		\textbf{Postcondición}\strut
	\end{minipage} & \begin{minipage}[t]{0.71\columnwidth}\raggedright\strut
		Se desactivan las posibilidades de \textit{Verificar lectura} y \textit{Guardar lectura}, se bloquean los botones.\strut
	\end{minipage}\tabularnewline
	\begin{minipage}[t]{0.23\columnwidth}\raggedright\strut
		\textbf{Excepciones}\strut
	\end{minipage} & \begin{minipage}[t]{0.71\columnwidth}\raggedright\strut
		A. No se consigue realizar la conexión con el radar.\\
		B. El fichero no tiene el formato correcto.\strut
	\end{minipage}\tabularnewline
	\begin{minipage}[t]{0.23\columnwidth}\raggedright\strut
		\textbf{Importancia}\strut
	\end{minipage} & \begin{minipage}[t]{0.71\columnwidth}\raggedright\strut
		Alta\strut
	\end{minipage}\tabularnewline
	\bottomrule
	\caption{CU-03 Clasificación de datos.}
\end{longtable}

\newpage

\begin{longtable}[H]{@{}ll@{}}
	\toprule
	\begin{minipage}[b]{0.23\columnwidth}\raggedright\strut
		\textbf{CU-04}\strut
	\end{minipage} & \begin{minipage}[b]{0.71\columnwidth}\raggedright\strut
		\textbf{Guardar datos en fichero.}\strut
	\end{minipage}\tabularnewline
	\midrule
	\endhead  
	\begin{minipage}[t]{0.23\columnwidth}\raggedright\strut
		\textbf{Requisitos asociados}\strut
	\end{minipage} & \begin{minipage}[t]{0.71\columnwidth}\raggedright\strut
		RF-1, RF-3, RF-4, RF-5, RF-8\strut
	\end{minipage}\tabularnewline
	\begin{minipage}[t]{0.23\columnwidth}\raggedright\strut
		\textbf{Descripción}\strut
	\end{minipage} & \begin{minipage}[t]{0.71\columnwidth}\raggedright\strut
		Permite al usuario guardar los datos extraídos de la lectura por radar o de un archivo.\strut
	\end{minipage}\tabularnewline
	\begin{minipage}[t]{0.23\columnwidth}\raggedright\strut
		\textbf{Precondición}\strut
	\end{minipage} & \begin{minipage}[t]{0.71\columnwidth}\raggedright\strut
		El equipo debe estar conectado a la red de internet \\
		El usuario debe tener iniciada la aplicación.\\
		El radar debe estar conectado a la red eléctrica.\\
		El radar debe estar conectado a la misma red de internet que el equipo por \textit{Ethernet} o \textit{Wifi}\strut
	\end{minipage}\tabularnewline
	\begin{minipage}[t]{0.23\columnwidth}\raggedright\strut
		\textbf{Acciones}\strut
	\end{minipage} & \begin{minipage}[t]{0.71\columnwidth}\raggedright\strut
		\begin{enumerate}
			\def\labelenumi{\arabic{enumi}.}
			\tightlist
			\item
			(Opcional) El usuario pulsa en la barra de herramientas en lectura.
			\item
			A. El usuario hace clic en \textit{Leer archivo} o icono de fichero.\\
			B. El usuario hace clic en \textit{Iniciar lectura} o icono de radar.
			\item
			A. Se inicia la conexión con el radar. \\
			B. Se abre el explorador de archivos.
			\item
			A. Se arranca el servicio de lectura en el radar. \\
			B. El usuario selecciona el archivo deseado.
			\item
			Se obtienen los datos.
			\item
			Se iluminan los botones de \textit{Verificar lectura} y \textit{Gurdar lectura}
			\item 
			El usuario pulsa en \textit{Guardar lectura}
			\item
			Se abre el explorador de archivos.
			\item
			El usuario debe indicar el nombre del fichero.
			\item
			El usuario debe guardar el fichero.
		\end{enumerate}\strut
	\end{minipage}\tabularnewline
	\begin{minipage}[t]{0.23\columnwidth}\raggedright\strut
		\textbf{Postcondición}\strut
	\end{minipage} & \begin{minipage}[t]{0.71\columnwidth}\raggedright\strut
		Se desactivan las posibilidades de \textit{Verificar lectura} y \textit{Guardar lectura}, se bloquean los botones.\strut
	\end{minipage}\tabularnewline
	\begin{minipage}[t]{0.23\columnwidth}\raggedright\strut
		\textbf{Excepciones}\strut
	\end{minipage} & \begin{minipage}[t]{0.71\columnwidth}\raggedright\strut
		A. No se consigue realizar la conexión con el radar.\\
		B. El fichero no tiene el formato correcto.\strut
	\end{minipage}\tabularnewline
	\begin{minipage}[t]{0.23\columnwidth}\raggedright\strut
		\textbf{Importancia}\strut
	\end{minipage} & \begin{minipage}[t]{0.71\columnwidth}\raggedright\strut
		Alta\strut
	\end{minipage}\tabularnewline
	\bottomrule
	\caption{CU-04 Guardar datos en fichero.}
\end{longtable}

\newpage

\begin{longtable}[H]{@{}ll@{}}
	\toprule
	\begin{minipage}[b]{0.23\columnwidth}\raggedright\strut
		\textbf{CU-05}\strut
	\end{minipage} & \begin{minipage}[b]{0.71\columnwidth}\raggedright\strut
		\textbf{Modificar dimensiones interfaz.}\strut
	\end{minipage}\tabularnewline
	\midrule
	\endhead  
	\begin{minipage}[t]{0.23\columnwidth}\raggedright\strut
		\textbf{Requisitos asociados}\strut
	\end{minipage} & \begin{minipage}[t]{0.71\columnwidth}\raggedright\strut
		RF-6, RF-7\strut
	\end{minipage}\tabularnewline
	\begin{minipage}[t]{0.23\columnwidth}\raggedright\strut
		\textbf{Descripción}\strut
	\end{minipage} & \begin{minipage}[t]{0.71\columnwidth}\raggedright\strut
		Permite al usuario modificar las dimensiones de la interfaz o minimizarla.\strut
	\end{minipage}\tabularnewline
	\begin{minipage}[t]{0.23\columnwidth}\raggedright\strut
		\textbf{Precondición}\strut
	\end{minipage} & \begin{minipage}[t]{0.71\columnwidth}\raggedright\strut
		El usuario debe tener iniciada la aplicación.\strut
	\end{minipage}\tabularnewline
	\begin{minipage}[t]{0.23\columnwidth}\raggedright\strut
		\textbf{Acciones}\strut
	\end{minipage} & \begin{minipage}[t]{0.71\columnwidth}\raggedright\strut
		\begin{enumerate}
			\def\labelenumi{\arabic{enumi}.}
			\tightlist
			\item
			A. EL usuario desplaza los bordes de la interfaz.\\
			B. El usuario pulsa en el icono de minimizar.
			\item 
			El usuario establece la dimensión final.
		\end{enumerate}\strut
	\end{minipage}\tabularnewline
	\begin{minipage}[t]{0.23\columnwidth}\raggedright\strut
		\textbf{Postcondición}\strut
	\end{minipage} & \begin{minipage}[t]{0.71\columnwidth}\raggedright\strut
		-\strut
	\end{minipage}\tabularnewline
	\begin{minipage}[t]{0.23\columnwidth}\raggedright\strut
		\textbf{Excepciones}\strut
	\end{minipage} & \begin{minipage}[t]{0.71\columnwidth}\raggedright\strut
		-\strut
	\end{minipage}\tabularnewline
	\begin{minipage}[t]{0.23\columnwidth}\raggedright\strut
		\textbf{Importancia}\strut
	\end{minipage} & \begin{minipage}[t]{0.71\columnwidth}\raggedright\strut
		Media\strut
	\end{minipage}\tabularnewline
	\bottomrule
	\caption{CU-05 Modificar dimensiones interfaz.}
\end{longtable}