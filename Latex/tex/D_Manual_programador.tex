\apendice{Documentación técnica de programación}

\section{Introducción}
En este anexo se detalla la estructura de directorios del proyecto y los conceptos necesarios para la programación del sistema así como para su instalación y ejecución. 

\section{Estructura de directorios}
\begin{itemize}
	\tightlist
	\item[•] \textbf{/}: raíz el proyecto, fichero de licencia, modelo clasificador \textit{k-NN}, \textit{interfaz RadarWave.py}, las librerias necesarias \textit{requeriments.txt} y archivo comprimido \textit{$RadarWave\_v1.0.zip$} con la primera versión de la aplicación.
	\item[•] \textbf{/acconeer-python-exploration/}: esta carpeta incluye la librería e interfaz de \textit{Acconeer}.
	\item[•] \textbf{/iconos/}: contiene los iconos de la interfaz.
	\item[•] \textbf{/Latex/}: documentación del proyecto memoria y anexos.
	\item[•] \textbf{/Latex/img/}: imágenes utilizadas en la documentación.
	\item[•] \textbf{/Latex/ltx/}: documentación en formato \LaTeX.
	\item[•] \textbf{/scripts/}: contiene los ficheros que se han utilizado para leer los materiales, crear las particiones de datos, los conjuntos de entrenamiento y test, todos los modelos clasificadores utilizados en el proyecto.
	\item[•] \textbf{/scripts/iconos/}: contiene los iconos de la interfaz.
	\item[•] \textbf{/scripts/Materiales/}: contiene todas las lecturas de los materiales.
	\item[•] \textbf{/scripts/Materiales/Fotos/}: fotos que identifican los objetos leídos.
\end{itemize}

\section{Manual del programador}

Los entornos para el desarrollo del código durante la realización de proyecto han sido las siguientes:
\begin{itemize}
\item \textit{Jupyter}
\item \textit{Colab}
\end{itemize}

\subsection{Entorno de programación}
El proyecto se ha desarrollado en los cuadernos de \textit{Jupyter Notebook} y \textit{Colab} mediante código y librerías \textit{Python}.
\imagen{jupyter}{Entorno Jupyter.}

Se ha decidido utilizar este entorno debido a que \textit{Jupyter} o \textit{Colab} siempre será 100\% de código abierto, de uso gratuito para todos. Es un entorno informático interactivo basado en la web para crear documentos de \textit{Jupyter Notebook}.

Además este entorno ya resultaba familiar debido a su uso en la Universidad de Burgos.

\section{Desarrollo del código}

En este apartado se muestra el desarrollo creado para la aplicación.

\subsection{Transformación de las características}

Una vez extraídas las características de los materiales se han realizado una serie de transformaciones en las lecturas, estas devuelven datos mostrados como números reales.

Se ha creado una función que a partir del nombre del fichero \textit{npy} cargue los datos, los redimensione y los devuelva.

Función $get\_data$ devuelve un array de 2 dimensiones (2D) al que llamaremos $datos\_bruto$.  

Los test de esta función son: 
\begin{itemize}
\item Comprobar que devuelve un array 2D
\item Comprobar que la segunda dimensión tiene tamaño 291
\item Comprobar que la matriz tiene una parte real y que tiene una parte imaginaria
\end{itemize}

\imagen{get_data}{Función $get\_data$}

Una función \textit{$get\_modulo\_fase$} que a partir del array 2D anterior, obtenga un array 2D, con el doble de anchura. Por ejemplo, si \textit{$datos\_bruto$} es de $300 x 291$, $modulo\_fase$ será de $600 x 291$. 

Esa función obtendrá el módulo del número complejo y la fase del número complejo. El módulo será una matriz de $300 x 291$ y la fase también.  Será necesario concatenarlas "horizontalmente". 

\imagen{get_modulofase}{Función $get\_modulo\_fase$}


A partir de los datos del modulo y la fase se obtiene la media. La comprobación sería que devuelva un array de 1 dimensión de tamaño $582 (291 x 2)$.

\imagen{get_media}{Función $get\_media$}

Finalmente se debían crear las particiones para el entrenamiento del modelo. Para ello se crea una función encargada de obtener las siguientes particiones separadas en entrenamiento ($part\_train$) y test ($part\_test$).

\begin{verbatim}
Partición 1
Entrenamiento: Carton Vista2-Vista10 + Plastico Vista2-Vista10 
				+ Cristal Vista2-Vista10
Test: Carton Vista1 + Plastico Vista1 + Cristal Vista1

Partición 2
Entrenamiento: Carton Vista1,Vista3-Vista10 + Plastico Vista1,Vista3-Vista10 
				+ Cristal Vista1,Vista3-Vista10
Test: Carton Vista2 + Plastico Vista2 + Cristal Vista2

...

Partición 10
Entrenamiento: Carton Vista1-Vista9 + Plastico Vista1-Vista9 
				+ Cristal Vista1-Vista9
Test: Carton Vista10 + Plastico Vista10 + Cristal Vista10

\end{verbatim}

Testamos esta función:
\begin{itemize}
\item Sumando el número de filas de test1 y train1 sea igual que sumando el número de filas de test2 y train2.
\item Se comprueba el tamaño de varias vistas igual a 582.
\end{itemize}
  

Dentro del repositorio de GitHub del proyecto hay un fichero llamado \href{https://github.com/mecyc/TFG_RADAR_60GHZ/blob/main/scripts/GenerarParticiones.ipynb}{\textit{GenerarParticiones.ipynb}} que incluye el desarrollo de todas estas funciones así como comentarios interesantes.

Una vez obtenidos los datos de las lecturas se comienza con la creación de varios modelos clasificadores, estos son:
\begin{itemize}
\item \href{https://github.com/mecyc/TFG_RADAR_60GHZ/blob/main/scripts/RandomForest.ipynb}{\textit{Random Forest}}
\item \href{https://github.com/mecyc/TFG_RADAR_60GHZ/blob/main/scripts/KNN.ipynb}{\textit{KNN}}
\item \href{https://github.com/mecyc/TFG_RADAR_60GHZ/blob/main/scripts/GridSearchCV\%20(SVM).ipynb}{\textit{SVM}}
\item \href{https://github.com/mecyc/TFG_RADAR_60GHZ/blob/main/scripts/Auto-Sklearn.ipynb}{\textit{auto-sklearn}}
\item \href{https://github.com/mecyc/TFG_RADAR_60GHZ/blob/main/scripts/TabPFN.ipynb}{\textit{TabPFN}}
\end{itemize}

Se ha decidido optar por \textit{KNN} para generar el modelo que utilizará la interfaz desarrollada. Esta decisión se debe a la falta de compatibilidad de \textit{auto-sklearn} y \textit{TabPFN} (basado en \textit{auto-sklearn}) con el sistema operativo \textit{Windows}. Actualmente \textit{auto-sklearn} es compatible con sistemas como \textit{Linux} y el desarrollo creado se quiere utilizar tanto en \textit{Windows} como en \textit{Linux}.

Finalmente, con el clasificador escogido creamos el modelo que emplea la aplicación a la hora de clasificar nuevos objetos leídos por el radar.

\subsection{Interfaz}

El desarrollo del código de la interfaz lo podemos encontrar en el cuaderno \href{https://github.com/mecyc/TFG_RADAR_60GHZ/blob/main/scripts/RadarWave.ipynb}{\textit{RadarWave.ipynb}}.

Para la generación de la interfaz se ha utilizado la librería \textit{Tkinter}, es una librería dedicada generalmente para el desarrollo de interfaces.

La aplicación generada para el proyecto ha sido llamada \textit{\textbf{RadarWave}}, debe su nombre a las ondas\footnote{Onda en inglés es wave.} generadas por los radares.

Para el desarrollo del código se ha decidido utilizar el patrón de diseño \textit{Facade}\cite{Facade} o \textit{Fachada}, utilizado habitualmente en aplicaciones escritas en \textit{Python}. Es de especial utilidad al trabajar con bibliotecas y \textit{API}\footnote{Interfaz de programación de aplicaciones.} complejas.


En la aplicación desarrollada se ha decidido crear una clase fachada y dos clases subsistemas.
Las clases son:
\begin{itemize}
\item[•] \textbf{FacadeRadarWave}: Clase \textit{Fachada} que implementa la interfaz de la aplicación.
\item[•] \textbf{TratamientoDatos}: Clase que comprende las funciones para el tratamiento de datos escogidos para su clasificación.
\item[•] \textbf{Verificacion}: Clase empleada para realizar verificaciones básicas en el inicio de la aplicación.
\end{itemize}

\textbf{Se ha decidido no incorporar el desarrollo del código de la interfaz en los anexos, se cree que es más comprensible su lectura y razonamiento dentro del cuaderno de \textit{Jupyter} \href{https://github.com/mecyc/TFG_RADAR_60GHZ/blob/main/scripts/RadarWave.ipynb}{\textit{RadarWave.ipynb}}.}

\section{Compilación, instalación y ejecución del proyecto}
\subsection{Compilación}
Al ser una aplicación \textit{beta} el archivo donde se encuentra desarrollada la interfaz \textit{RadarWave.ipynb} se ha exportado en formato \textit{.py} de \textit{Python}. Además se ha creado un archivo llamado \textit{requirements.txt} que será utilizado para instalar las librerías necesarias en nuestro equipo.

Todos los archivos necesarios se han comprimido para su exportación en el fichero \textit{$RadarWave\_1.0.zip$}

\subsection{Instalación en \textit{Windows}}

La siguiente instalación se ha realizado en un equipo con las siguientes características:
\begin{itemize}
\item[•] \textbf{SISTEMA OPERATIVO:} \textit{Microsoft Windows 10 Pro}
\item[•] \textbf{PROCESADOR:} \textit{Intel(R) Core(TM) i7-4720HQ CPU @ 2.60GHz}
\item[•] \textbf{MEMORIA RAM:} 12 GB
\item[•] \textbf{TARJETA GRÁFICA:} \textit{NVIDIA GeForce 930M}
\end{itemize}

En realidad la aplicación \textit{RadarWave} no hace falta que se instale, pero si necesitamos instalar las librerías utilizadas por el programa.

Antes de proceder con la instalación de las librerías vamos a verificar si tenemos instalado \textit{Python}, de no ser así procedemos a su instalación.
\begin{itemize}
\item Descargamos el instalador desde la web de \href{https://www.python.org/downloads/}{\textit{Python}}.
\item Ejecutamos el archivo descargado.
\item En la pantalla de de inicio de la instalación marcamos la opción de añadir \textit{Python} al \textit{PATH}, si no luego se tendrá que hacer manualmente.
\item Pulsamos en instalar ahora (\textit{install now})

\imagen{path}{Instalación de \textit{Python}.}

\end{itemize}


Una vez instalado \textit{Python} lo primero que haremos será descargar y descomprimir el fichero \textit{$RadarWave\_v1.0.zip$}, se puede descargar desde \url{https://github.com/mecyc/TFG_RADAR_60GHZ/blob/main/RadarWave_v1.0.zip}

Procedemos a instalar las librerías, están incluidas en el archivo \textit{requirements.txt}. Primero instalamos \textit{setuptools} que facilita el empaquetado de proyectos de \textit{Python} y a continuación el resto, para ello debemos abrir la consola y lanzar las siguientes instrucciones.

\begin{verbatim}
python -m pip install -U --user setuptools wheel
python -m pip install -U --user -r requirements.txt
\end{verbatim}

\imagen{requirements}{Instalar librerías.}

Finalmente se debe instalar la librería de \textit{Acconeer}, para ello descargamos del repositorio del TFG la carpeta \href{https://github.com/mecyc/TFG_RADAR_60GHZ/tree/main/acconeer-python-exploration}{\textit{acconer-python-exploration}}. Una vez descargada abrimos el terminal dentro de la carpeta y ejecutamos la siguiente orden.\cite{Acconeer2021}

\begin{verbatim}
python -m pip install -U --user .
\end{verbatim}

\imagen{instalar_libacconer}{Instalar acconeer.}

Una vez instaladas las librerías procedemos a ejecutar el programa haciendo doble clic en el fichero \textit{RadarWave.py} donde está el programa.

Programa ejecutado y probado en \textit{Windows}:

\imagen{radarwaveWindows}{\textit{RadarWave} en \textit{Windows}.}

\subsection{Instalación en \textit{Linux}}

La siguiente instalación se ha realizado en un equipo virtualizado en \textit{Oracle VM VirtualBox 6.1} con las siguientes características:
\begin{itemize}
\item[•] \textbf{SISTEMA OPERATIVO:} \textit{Ubuntu 22.04 LTS}
\item[•] \textbf{PROCESADOR:} \textit{Intel(R) Core(TM) i7-4720HQ CPU @ 2.60GHz}
\item[•] \textbf{MEMORIA RAM:} 4 GB
\end{itemize}

En realidad la aplicación \textit{RadarWave} no hace falta que se instale, pero si necesitamos instalar las librerías utilizadas por el programa.

El sistema operativo que utilizamos es \textit{Ubuntu}, una distribución de {Linux}, por lo que \textit{Python} está integrado por defecto y no hace falta su instalación. Se recomienda tener actualizado \textit{Python} a la última versión.

Antes de instalar las librerías necesarias instalamos \textit{pip}, es un sistema de gestión de paquetes que nos ayuda a instalar las librerías, y \textit{setuptools}, facilita el empaquetado de proyectos de \textit{Python}.

\begin{verbatim}
sudo apt install python3-pip
pip3 install setuptools-rust
\end{verbatim}

Una vez terminada la instalación procedemos a descargar y descomprimir el fichero \textit{$RadarWave\_v1.0.zip$}, se puede descargar desde \url{https://github.com/mecyc/TFG_RADAR_60GHZ/blob/main/RadarWave_v1.0.zip}

A continuación se debe instalar las librerías que usa la interfaz desarrollada. Indicar que para ejecutar el siguiente comando nos debemos situar en la dirección o \textit{PATH} donde se encuentra el archivo \textit{requirements.txt}, lo encontramos dentro del fichero que acabamos de descomprimir.

El comando sería el siguiente:
\begin{verbatim}
pip3 install -r requirements.txt
\end{verbatim}

Si existiera algún problema al instalar la librería de \textit{Tkinter} (tk) se deben lanzar los siguientes comandos:
\begin{verbatim}
sudo apt-get install python3-tk
sudo apt-get install python3-pil python3-pil.imagetk
\end{verbatim}

Finalmente se debe instalar la librería de \textit{Acconeer}, para ello descargamos del repositorio del TFG la carpeta \href{https://github.com/mecyc/TFG_RADAR_60GHZ/tree/main/acconeer-python-exploration}{\textit{acconer-python-exploration}}. Una vez descargada abrimos el terminal dentro de la carpeta y ejecutamos la siguiente orden.

\begin{verbatim}
python3 setup.py install
\end{verbatim}

Si queremos ejecutar \textit{RadarWave} en \textit{Ubuntu} deber ser lanzando el siguiente comando en la misma ruta del archivo \textit{RadarWave.py}.

\begin{verbatim}
python3 RadarWave.py
\end{verbatim}

Programa ejecutado y probado en \textit{Ubuntu}:

\imagen{radarwaveUbuntu}{\textit{RadarWave} en \textit{Ubuntu}.}

\section{Pruebas del sistema}
Durante el desarrollo del código del programa se fueron implementando un conjunto de pruebas para comprobar el correcto tratamiento de los datos de entrenamiento.

Estas pruebas las podemos encontrar en el fichero \textit{GenerarParticiones.ipynb}, aquí se crearon varias pruebas para comprobar que los resultados eran los correctos. En la sección \textbf{\textit{Desarrollo del código}} del anexo actual hemos hablado de las pruebas o test que se han formulado.

Los test realizados ha sido:

\begin{verbatim}
#comprobar que devuelve un array 2D
assert(len(get_data('Materiales\C01_V01.npy').shape) == 2) 

#comprobar que la segunda dimensión tiene tamaño 291
assert(get_data('Materiales\C01_V01.npy').shape[1] == 291) 

#comprobar que la matriz tiene una parte real
assert((get_data('Materiales\C01_V01.npy').real.mean) != None) 

#comprobar que la matriz tiene una parte imaginaria
assert((get_data('Materiales\C01_V01.npy').imag.mean) != None) 

assert(get_media(modulo_fase).shape[0] == 582) #Dimensión de tamaño = 582

assert(len(get_media(modulo_fase).shape) == 1) #Array de 1 dimensión

#Train1 60 + Test1 30 = Train2 60 + Test2 30
assert(len(train[0]) + len(test[0])) == (len(train[1]) + len(test[1]))
 
#Train1 60 + Test1 30 = Train10 60 + Test10 30
assert(len(train[0]) + len(test[0])) == (len(train[9]) + len(test[9]))

#582 son las características de una vista
assert(len(train[0][0]) == 582)
 
#Las vistas tienen todas un tamaño de 582 (291 x 2)(modulo x fase)
assert(len(train[0][0]) == len(train[0][9]) == len(test[5][7])) 
\end{verbatim}