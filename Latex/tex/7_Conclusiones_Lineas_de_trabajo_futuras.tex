\capitulo{7}{Conclusiones y Líneas de trabajo futuras}

\section{Conclusiones}

En este trabajo se exploraron las posibilidades de utilizar un radar de ondas de 60GHz para la clasificación de diferentes tipos de materiales. Se descubrió que extrayendo características mediante el radar y combinando esto con un clasificador Random Forest es posible realizar una clasificación  para distinguir tres tipos de materiales comunes en la vida diaria como son PLÁSTICO, CRISTAL y CARTÓN con una precisión del 99,89\% usando validación cruzada. 

Mediante una aplicación en fase Beta se ha conseguido realizar la lectura de las características de varios materiales y su correcta identificación.

Una posible aplicación sería automatizar la diferenciación de los tipos de embalajes en una empresa de envíos para su correcto reparto evitando los golpes.

\section{Líneas de trabajo futuras}

Posibles líneas de trabajo futuras relativas al uso del radar:

\begin{itemize}
\item[•] Combinar el radar utilizado de Acconeer con un sensor radar del fabricante Infineon Technologies, posiblemente colocados en dos posiciones del habitáculo de lectura (un radar superior y otro lateral).
\item[•] Refactorización del código actual para su uso en otros sensores.
\item[•] Aplicar otros clasificadores.
\item[•]Añadir otros estadísticos.
\item[•] Automatizar el proceso de creación y entrenamiento del clasificador mediante una base de datos, así como la ampliación de la capacidad de reconocer materiales nuevos.
\end{itemize}

